%++++++++++++++++++++++++++++++++++++++++
% Don't modify this section unless you know what you're doing!
\documentclass[letterpaper,12pt]{article}
\usepackage{tabularx} % extra features for tabular environment
\usepackage{amsmath}  % improve math presentation
\usepackage{authblk}  % for authors
\usepackage{graphicx} % takes care of graphic including machinery
\usepackage{color}
\usepackage[margin=1in,letterpaper]{geometry} % decreases margins
\usepackage{cite} % takes care of citations
\usepackage[final]{hyperref} % adds hyper links inside the generated pdf file
\hypersetup{
	colorlinks=true,       % false: boxed links; true: colored links
	linkcolor=blue,        % color of internal links
	citecolor=blue,        % color of links to bibliography
	filecolor=magenta,     % color of file links
	urlcolor=blue
}
\definecolor{orange}{RGB}{255,127,0}
\newenvironment{centerverbatim}{%
  \par
  \centering
  \varwidth{\linewidth}%
  \verbatim
}{%
  \endverbatim
  \endvarwidth
  \par
}
%++++++++++++++++++++++++++++++++++++++++


\begin{document}

\title{\textbf{HERACLES and DJANGOH : Event Generation of $lp$ Interactions including Radiative Processes}}
\author[1,2]{\textbf{N. Pierre}}
\author[3]{\textbf{H. Spiesberger}}
\affil[1]{\textit{CEA IRFU/SPhN Saclay, 91191 Gif-sur-Yvette, France}}
\affil[2]{\textit{Universit\"at Mainz, Institut f\"ur Kernphysik, 55099 Mainz, Germany}}
\affil[3]{\textit{Universit\"at Mainz, Institut f\"ur Physik, 55099 Mainz, Germany}}
\maketitle

\begin{abstract}
  Monte Carlo event simulation of neutral and charged current $lp$ interactions with
  with the event generator HERACLES and DJANGOH is described. The emphasis is put
  on the inclusion of radiative corrections, comprising single photon emission from
  the lepton and the quark line as well as self-energy corrections and the complete
  set of one-loop weak corrections. The event generator HERACLES optionnally treats
  the $lp$ scattering either by means of structure function parametrization of on the
  basis of parton distribution functions in the framework of the quark-parton model.
  The background from radiative elastic scattering $lp \rightarrow lp\gamma$ is also
  included. The event generator DJANGOH simulates deep inelastic lepton-proton
  scattering including both QED and QCD radiative effects. DJANGOH is an interface
  of the Monte Carlo programs HERACLES and LEPTO. Alternatively, parton cascades
  can be genrated also with the help of ARIADNE. The LUND string fragmentation as
  implemented in the event simulation program JETSET is used to obtain the complete
  hadro-nic final state, provided there is enough hadronic mass. Otherwise SOPHIA is
  used to generate low-mass hadronic final states. TDJANGOH, the C++ interface around
  HERACLES and DJANGOH is also described.
\end{abstract}

\newpage

\section{Introduction}

The very important physical effect has applications to astronomy, nuclear physics, condensed matter, and more.



\newpage
\section{DJANGOH : a Monte-Carlo generator with radiative corrections}

\subsection{Presentation of DJANGOH}
\textbf{What is DJANGOH ?}
\begin{itemize}
\item DJANGOH\cite{DJANGOH} is at first a Monte-Carlo event simulation tool for neutral and charged current
$ep$ interactions at HERA with the event generators HERACLES and DJANGO6.
\item The emphasis is put on the inclusion of QED radiative corrections
(single photon emission from the lepton or the quark line, self energy correction, complete
set of one-loop weak corrections). The background from radiative elastic scattering
$\mu p\rightarrow \mu p\gamma$ is also included.
\item HERACLES is treating the $lp$ scattering by means of structure function parametrizations
or parton distribution functions in the quark-parton model framework.
\item DJANGO6 is simulating deep inelastic scattering including both QED and QCD radiative effects.
\item DJANGOH is an interface to LEPTO\cite{LEPTO}, ARIADNE\cite{ARIADNE} (for parton cascades), PYTHIA\cite{PYTHIA6} (LUND
string fragmentation in JETSET\cite{JETSET} for hadronic final state) and SOPHIA\cite{SOPHIA} (for low-mass
hadronic final states).
\end{itemize}
\textbf{What can one do with DJANGOH ?}
\begin{itemize}
\item Generation of $lp$ scattering with and without fragmentation for the final state with radiative events.
\item Calculation of cross-sections (radiative, born)
\item Calculation of radiative correction factors (inclusive, semi-inclusive)
\item Use it as an event generator in a Monte-Carlo chain
\end{itemize}

\subsection{Forewords : Technical description of DJANGOH}

The computational procedures applied in DJANGOH are based on the methods used in AXO\cite{AXO}
library for Monte-Carlo integration and event generation. AXO relies on the Monte-Carlo
integration algorithm VEGAS\cite{VEGAS}. The computation is made in this order :

\begin{itemize}
\item Integration of the different contributions : partial cross-sections are determined
according to the defined phase-space region. They give the relative weight of the
corresponding contribution in the final step of event sampling. Moreover, the integration
procedure supplies information for the construction of the distribution function
applied for an event generation.
\item Estimation of the local maxima of the distribution function in a predefined number
of hypercubes.
\item According to the partial cross-sections that were calculated, events are generated
randomly from the individual contributions. HERACLES is only taking care of the scattered
lepton and the potential radiative photon. DJANGO is simulating the QCD effect and generates
the hadronic part of the event.
\end{itemize}

\section{Upgrading DJANGOH : paradigm shift and TDJANGOH interface}

An upgrade to the original DJANGOH is the possibility to use different input energies for
the incoming lepton for mutiple event generation. It was before only allowed to
specify one input energy at the launch of the program. DJANGOH is now capable to take
into account a new beam energy at each new event.
Nonetheless, using different input energies for event generation is causing a problem :
the cross-sections that are needed for event generation are depending on this input energy.
The naive way would be to recompute the cross-section for each event but this solution
takes much time due to the computation of the cross-section being the slowest part of
the generation.

A solution to circumvent this problem is the use
of a grid of cross-sections. This grid is initialized after a rough specification of the
type of dispersion in energy of the considered beam.
Basically, the grid needs to have
a mean energy and the standard deviation of energy to this mean energy, as well as the
number of bins in the grid. Then for each bin, the energy of the center of the bin is taken
and the cross-section corresponding to this energy is computed. The narrower the bins are
the more accurate the cross-section is for the considered bin. As shown in Fig. \ref{fig:gridxs}, with a
mean energy of 160 GeV, a distribution width of 20 GeV and 20 bins, the grid is giving an
accurate map of the cross-sections. Though the difference of cross-section is not very large (13.9 nB, 2\% of
the total cross-section), it has to be taken into account for a proper event generation.

The user can communicate with the program by setting initial condition via codewords.
Principal useful codewords from the HERACLES part are :
\begin{itemize}
\item \textbf{EL-BEAM : Quantities defining the properties of the lepton beam.}
\begin{description}
\item[polari] : Polarisation of the beam
\item[beam] : Particle of the beam ($e^+$,$e^-$,$\mu^+$,$\mu^-$)
\end{description}
\item \textbf{KINEM-CUTS : Definition of the kinematical cuts. These cuts are applied to the
leptonic variables $x_l$, $y_l$, $Q^2_l$, defined by the momentum of the final-state
lepton and the mass of the hadron.}
\begin{description}
\item[icut] : Determine the number of active cuts (D=3)
\begin{description}
\item=1 : cuts in $x_{l}$ and lower cut in $Q^2_l$
\item=2 : cuts in $x_{l}$, lower cut in $Q^2_l$ and lower cut in $W_h$,
\item=3 : cuts in $x_l$, $y_l$, $Q^2_l$ and $W_h$)
\end{description}
\item[ixmin] : Minimal $x_l$
\item[ixmax] : Maximal $x_l$
\item[iymin] : Minimal $y_l$
\item[iymax] : Maximal $y_l$
\item[iq2min] : Minimal $Q^2_l$
\item[iq2max] : Maximal $Q^2_l$
\item[iwmin] : Minimal $W_h$
\end{description}
\item \textbf{GSW-PARAM : Monitoring the definition of the electroweak parameters and the
inclusion of the different virtual corrections.}
\begin{description}
\item[lparin1] : (D=2)
\begin{description}
\item=1 : Electroweak parameters set with fixed values forthe boson masses $M_W$, $M_Z$)
\item=2 : Electroweak parameters calculated from fixed $M_Z$, $G_\mu$
\end{description}
\item[lparin2] : (D=1)
\begin{description}
\item=0 : Only Born cross section without electromagnetic or weak corrections is integrated
\item=1 : Born cross section including corrections is determined by the values of lparin[3] to lparin[11]
\end{description}
\item[lparin3] : Flag for inclusion of higher order contributions (D=3)
\begin{description}
\item=0 : No higher order corrections
\item$\geq1$ : Terms of $O(\alpha^2m_t^4)$ included
\item$\geq2$ : Terms of $O(\alpha\alpha_sm_t^4)$ included
\item$\geq3$ : Running $O(Q^2)$ is used for the radiative cross section
\end{description}
\item[lparin4] : Leptonic QED corrections (D=1)
\item[lparin5] : Quarkonic QED correction (D=0)
\item[lparin6] : Lepton-quark interference (D=0)
\newpage
\item[lparin7] : Fermionic contribution to the photon self-energy $\Sigma^\gamma$ (D=2)
\begin{description}
\item=0 : Not included
\item=1 : Parametrization with the help of quark masses
\item=2 : Parametrization from ref 20
\end{description}
\item[lparin8] : Fermionic contribution to the $\gamma - Z$ mixing (D=0)
\item[lparin9] : Fermionic contribution to the self-energy of the Z boson (D=0)
\item[lparin10] : Fermionic contribution to the self-energy of the W boson (D=0)
\item[lparin11] : Purely weak contributions to the self-energy, vertex corrections and boxes (D=0)
\end{description}
\item \textbf{GD-OPT : Definition of the size of the cross-section grid.}
\begin{description}
\item[gdmean] : Central value of the energy distribution
\item[gdsddv] : Width of the energy distribution
\item[gdsize] : Number of bins in the grid
\end{description}
\item \textbf{EGAM-MIN : Definition of a lower cutoff energy for bremsstrahlung photons (in GeV).}
\begin{description}
\item[egam] : Value of the cutoff
\end{description}
\item \textbf{INT-OPT-NC : Defines the contributions to the neutral current interactions for which
the integrated cross-section is asked to be calculated.}
\begin{description}
\item[inc2] : Integration for the non-radiative contribution (Born term including virtual and soft corrections) (D=1)
\item[inc31] : Number of iteration for integration of the contribution from initial state leptonic bremsstrahlung by VEGAS (D=18)
[NB : There are subtle difference at different scale of iteration (<100, <200). These are explicited in DJANGOH manual. This remark
is valid for all other integration that does iteration.]
\item[inc32] : Number of iteration for integration of the contribution from final state leptonic bremsstrahlung by VEGAS (D=18)
\item[inc33] : Number of iteration for integration of the Compton contribution by VEGAS (D=18)
\item[inc34] : Number of iteration for integration of the contribution from radiation from the quark line by VEGAS (not active in DJANGOH, D=0)
\item[iel2] : Integration of the elastic scattering $lp\rightarrow lp$ (D=1)
\item[iel31] : Number of iteration for integration of the quasielastic tail $lp\rightarrow lp\gamma$ from initial state leptonic bremsstrahlung by VEGAS (D=18)
\item[iel32] : Number of iteration for integration of the quasielastic tail $lp\rightarrow lp\gamma$ from final state leptonic bremsstrahlung by VEGAS (D=18)
\item[iel33] : Number of iteration for integration of the quasielastic tail $lp\rightarrow lp\gamma$ from the Compton contribution by VEGAS (D=18)
\end{description}
\item \textbf{INT-POINTS : Upper limit for the number of integration points used by VEGAS (D=10000).}
\item \textbf{SAM-OPT-NC : Monitors the inclusion of individual contributions to the neutral current
cross-section for event sampling.}
\begin{description}
\item[isnc2] : Non-radiative contribution (D=1)
\item[isnc31] : Initial state leptonic bremsstrahlung (D=1)
\item[isnc32] : Final state leptonic bremsstrahlung (D=1)
\item[isnc33] : Compton contribution (D=1)
\item[isnc34] : Quarkonic radiation (D=0)
\item[isel2] : Elastic scattering $lp\rightarrow lp$ (D=1)
\newpage
\item[isel31] : Initial state leptonic radiation for $lp\rightarrow lp\gamma$ (D=1)
\item[isel32] : Final state leptonic radiation for $lp\rightarrow lp\gamma$ (D=1)
\item[isel33] : Compton contribution for $lp\rightarrow lp\gamma$ (D=1)
\end{description}
\item \textbf{NUCLEUS : Type and energy of the nucleus.}
\begin{description}
\item[epro] : Energy of the nucleus
\item[hpolar] : Polarization of the nucleus
\item[hna] : Number of nucleons in the nucleus
\item[hnz] : Number of protons in the nucleus
\end{description}
\item \textbf{STRUCTFUNC : Defines the parametrization of the parton densities or structure functions
applied.}
\begin{description}
\item[ilib] : (D=2, icode=10150)
\begin{description}
\item=1 : to choose PDFs from PYTHIA. For more infos on the different values for icode, please
refer to DJANGOH manual.
\item=2 : to use PDFLIB\cite{PDFLIB}. Same remark than previously.
\end{description}
\item[ilqmod] : The options with ilqmod$\geq2$ do not allow event generation (D=0)
\begin{description}
\item=0 : Use unmodified PDFs for all $Q^2$
\item=1 : Apply exponentially low $Q^2$ suppression factor to PDFs.
\item=2 : For $Q^2<6$ GeV$^2$, $F_1$ and $F_2$ parametrizations from Brasse.
\item=3 : For $Q^2<6$ GeV$^2$, $F_1$ and $F_2$ parametrizations from Abramowicz, Levy, Levin and Maor.
Brasse is used for $W<2$ GeV.
\item=4 : For $x<0.1$ and $\nu>10$ GeV, $F_1$ and $F_2$ parametrizations from Badelek and Kwiecinski.
For $x<0.1$ and $\nu<10$ GeV, Brasse and Stein parametrization is used.
\item=5 : For $Q^2<10$ GeV$^2$, $F_1$ and $F_2$ parametrizations from Donnachie-Landschoff together with
PDFs.
\item=6 : $F_1$, $F_2$ and $R$ parametrizations from TERAD.
\end{description}
\end{description}
\item \textbf{FLONG : Inclusion of the longitudinal structure function $F_L$.}
\begin{description}
\item[iflopt] = LQCD + LTM*10 + LHT*100, where LQCD, LTM, LHT specifies whether QCD contributions,
target mass effects and higher twists should be included. See manual of LEPTO 6.5.1 for more details. (D=111)
\item[parl111] : Accuracy for the integration needed to calculate $F_L$. (D=0.01)
\item[parl19] : $\kappa^2$, the scale parameter for higher-twist contributions in $F_L$. (D=0.03)
\end{description}
\item \textbf{ALFAS : Define the value of $\alpha_s$ needed for the inclusion of $F_L$.}
\begin{description}
\item[mst111] : Order of $\alpha_s$ evaluation (D=1)
\begin{description}
\item=0 : fixed $\alpha_s$ at the value of par111
\item=1 : first-order running $\alpha_s$
\item=2 : second-order running $\alpha_s$
\end{description}
\end{description}
\begin{description}
\item[mst115] : Treatment of the $\alpha_s$ singularity at $Q^2 \rightarrow 0$ (D=0)
\begin{description}
\item=0 : allow it to diverge like $1/ln(Q^2/\Lambda^2)$
\item=1 : soften the divergence to $1/ln(1+Q^2/\Lambda^2)$
\item=2 : freeze $Q^2$ evolution below $Q^2_0=4$ GeV$^2$
\end{description}
\end{description}
\item \textbf{NFLAVORS : Minimal and maximal number of flavor to be included in cross-section} (DMin=1,DMax=6)
\end{itemize}
\newpage
Principal useful codewords from the DJANGO part are :
\begin{itemize}
\item \textbf{FRAG : Activation of the fragmentation.}
\begin{description}
\item[lst7] : (D=1)
\begin{description}
\item=-1 : no generation of parton cascades and no fragmentation
\item=0 : event generation at the parton level, no hadronization
\item=1 : event generation including hadronization and decays of unstable particles
\end{description}
\end{description}
\item \textbf{CASCADES : Describes the part of the event simulation which is determined by
perturbative QCD.}
\begin{description}
\item[lst8] : (D=12)
\begin{description}
\item=0 : no QCD effects, no gluon radiation or boson-gluon fusion
\item=1 : including QCD processes (gluon radiation and boson-gluon fusion) according
to the first-order matrix elements
\item=2 : QCD parton cascade evolution of the initial and final quark
\item=3 : QCD parton cascade evolution of the initial quark only
\item=4 : QCD parton cascade evolution of the final quark only
\item=5 : QCD switched off but target treatment exactly as in parton cascade case
\item=9 : simulating QCD cascades in the colour dipole model (as implemented in ARIADNE)
\item=12-15 : as 2-5 but parton shower added to the event as obtained from first-order matrix elements
\end{description}
\end{description}
\item \textbf{MAX-VIRT : Maximal virtuality in parton cascades.}
\begin{description}
\item[lst9] : (D=5)
\begin{description}
\item=1 : $Q^2$
\item=2 : $W^2$
\item=3 : $W.Q$
\item=4 : $Q^2(1-x)$
\item=5 : $Q^2(1-x)max(1,ln(1/x))$
\item=6 : $x_0W^2$
\item=9 : $W^{4/3}$, similar as in the colour dipole model
\end{description}
\end{description}
\item \textbf{SOPHIA : Defines the upper limit for the hadronic mass below which SOPHIA is
called for the generation of the hadronic final state} (D=1.4)
\end{itemize}

\subsection{Implementing DJANGOH in TGEANT}

The original DJANGOH is a FORTRAN framework. The main problem I encountered is that
DJANGOH was supposed to be used in the Monte-Carlo simulation of the COMPASS apparatus,
TGEANT. As the whole simulation is coded in C++, the inclusion of a new generator inside
the chain is done via the integration of a plugin in C++ containing the generator.
Thus, in order to include DJANGOH into TGEANT, I had to modify DJANGOH so that it
can be used as a C++ plugin generator by TGEANT.

Fig. \ref{fig:evtgenTG} shows the principal way of event generation in TGEANT.

First, a particle is selected with the T4PrimaryGen class, that reads a given beamfile,
and pick a particle at random. Then the T4ProcessBackEnd class generates
the given process and creates the outgoing particles that will be detected afterwards
by the detectors. This T4ProcessBackEnd class is purely virtual and is replaced by
every generator class implemented in TGEANT.

\newpage
There are two ways to implement a generator in TGEANT :
\begin{itemize}
\item As an internal generator (Pythia and HEPGEN++ way) : a C++ interface to the
generator is needed, the conservation of $P(p,E)$ is perfect and it only needs a
beamfile for the primary generator.
\item As an external generator (Lepto) : the beamfile is read by the standalone
generator, the primary generation and process infos are stored inside a file and
TGEANT has to do an extrapolation to -9 meters as the primary generation was done
outside of it, occasioning $P(p,E)$ being not perfectly conserved.
\end{itemize}

The external implementation is quite complicated as three classes are needed in order
to use the generator in TGEANT. First, the beamfile has to be read by the generator
which then outputs a file containing the beamfile infos but in its own format.
Then a first class has to be dedicated to the reading of the beamfile, recovering
the information about the selected muon in the file. A second class takes care of
passing of information between the generator and TGEANT. The last class is the
class of the generator itself (Fig. \ref{fig:leptoex}).

\newpage
If you apply this method to DJANGOH, then you encounter several critical problems :
\begin{itemize}
\item A new file format for the converted beamfile has to be created.
\item DJANGOH has to be modified to allow backward propagation of the incoming muon.
\item The result of fragmentation (LUJETS) has to be recovered in a file.
\end{itemize}
This results in many file accesses and consequently it is a non-efficient way to
implement DJANGOH inside TGEANT.

The internal implementation recquires perhaps more work on the generator itself however
this solution is much more efficient. Here, only two class are needed. One is the
interface class that creates instances of DJANGOH that can be manipulated in any C++
environment. This class is a C++ interface that is handling the FORTRAN part of DJANGOH.
The other class is taking care of passing the information between the interface to the
generator and TGEANT (Fig. \ref{fig:pythiaex}).

This second method was used to implement DJANGOH inside TGEANT.

\subsection{TDJANGOH : a C++ interface to DJANGOH}

In order to create a C++ class (called TDJANGOH) that plays the role of an interface, I had to modify some
FORTRAN parts of DJANGOH, especially the input method. DJANGOH is working with an input
file where codewords with set values are specified in order to configure the generator.
This was not convenient for the idea of an interface. Thus I have drawn
correspondances between Common Blocks in FORTRAN and structures in C++ so that I can specify
values in the C++ structure and the change is repercuted in the FORTRAN code and vice-versa.
It is useful to specify the values for the input but also to recover the results of the
hadronization that are located in the LUJETS Common Block.
The idea is that within TGEANT the user specifies the input for DJANGOH, the interface
pass it to the generator and the interface recovers the results of the generator and gives
it to TGEANT.

\newpage

The principal and most useful methods that are included in the interface are :
\begin{itemize}
\item \textbf{{\color{orange}static} TDjangoh *Instance()} : Allow the creation of an instance of TDJANGOH. Only one
instance at a time is permitted.
\item \textbf{{\color{orange}void} ReadXMLFile(const string pFilename)} : Read an XML file in order to initialize the input
parameters of DJANGOH.
\item \textbf{{\color{orange}void} Initialize()} : Initialize the generator with previously defined values and computes cross-sections.
\item \textbf{{\color{orange}void} GenerateEvent()} : Generates an event. Always use after having called Initialize() !
\end{itemize}

Some unavoidable toolbox methods are also present :
\begin{itemize}
\item \textbf{{\color{orange}void} SetParticle(const char* pname)} : Modify the beam particle (useful to switch between $\mu^{+}$ and $\mu^{-}$)
\item \textbf{{\color{orange}void} Configure(float beam\_e, float pol)} : Modify the beam properties (energy, polarisation)
\item \textbf{{\color{orange}void} BornWOqelNC()} : Call for event generation only taking into account the Born cross-section without quasi-elastic.
\item \textbf{{\color{orange}void} RClepWOqelNC()} : Call for event generation taking into account the Born cross-section + ISR,FSR and Compton, without quasi-elastic.
\item \textbf{{\color{orange}void} EndRecap()} : Print a recap in TDjangoh\_out.dat at the end of the generation.
\item \textbf{{\color{orange}void} Clean\_File()} : Clean all the .dat files created by the generator.
\end{itemize}

For informations about the other methods that are available within the interface, please refer to the
doxyfile of TDJANGOH.

\newpage
\section{TDJANGOH : How to ?}

In order to get TDJANGOH running, recover a version in the following git repository :

In the test programs :
\begin{itemize}
\item test allows you to do full event generation. You have some options if you enter
some flags, do bin/test -h to have more infos about it.
\item xsgen allows you to compute cross-section rapidly, without any hadronisation in
order to gain time. Do bin/xsgen -h for more infos.
\end{itemize}

In the utilitaries :
\begin{itemize}
\item rc\_calc allows you to plot the inclusive radiative corrections after different kind of inputs. The most
convenient one is to use the results from xsgen and feed it directly with the -sigf option.
\item sirc allows you to plot the multiplicities for Born and Born+o($\alpha$) and see the comparison between
the two. The distribution are in $(x_{Bj},y,z)$ and $(x_{Bj},y,p_T)$
\item pdist allows you to plot various information about the radiative photons, like the energy distribution,
angle distribution, etc.
\item lowQ2 is a test code for plotting 3D plots of $F_2$ functions in $(x,Q^2)$. Its usefulness is limited,
but can be informative when comparing different parametrization of $F_2$.
\end{itemize}

The XML configuration files are located in the utils/ directory.


\section{Conclusions}
Here you briefly summarize your findings.

%++++++++++++++++++++++++++++++++++++++++
% References section will be created automatically
% with inclusion of "thebibliography" environment
% as it shown below. See text starting with line
% \begin{thebibliography}{99}
% Note: with this approach it is YOUR responsibility to put them in order
% of appearance.

\begin{thebibliography}{99}

% \bibitem{melissinos}
% A.~C. Melissinos and J. Napolitano, \textit{Experiments in Modern Physics},
% (Academic Press, New York, 2003).

\end{thebibliography}


\end{document}
