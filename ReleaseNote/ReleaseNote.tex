%++++++++++++++++++++++++++++++++++++++++
% Don't modify this section unless you know what you're doing!
\documentclass[letterpaper,12pt]{article}
\usepackage{tabularx} % extra features for tabular environment
\usepackage{amsmath}  % improve math presentation
\usepackage{authblk}  % for authors
\usepackage{graphicx} % takes care of graphic including machinery
\usepackage{color}
\usepackage[margin=1in,letterpaper]{geometry} % decreases margins
\usepackage{cite} % takes care of citations
\usepackage[final]{hyperref} % adds hyper links inside the generated pdf file
\hypersetup{
	colorlinks=true,       % false: boxed links; true: colored links
	linkcolor=blue,        % color of internal links
	citecolor=blue,        % color of links to bibliography
	filecolor=magenta,     % color of file links
	urlcolor=blue
}
\definecolor{orange}{RGB}{255,127,0}
\newenvironment{centerverbatim}{%
  \par
  \centering
  \varwidth{\linewidth}%
  \verbatim
}{%
  \endverbatim
  \endvarwidth
  \par
}
%++++++++++++++++++++++++++++++++++++++++


\begin{document}

\title{\textbf{Charged kaon multiplicities from muon deep inelastic scattering on lH$_2$ (2016 COMPASS data)}}
\author[1,2]{\textbf{N. Pierre}}
\affil[1]{\textit{CEA IRFU/SPhN Saclay, 91191 Gif-sur-Yvette, France}}
\affil[2]{\textit{Universit\"at Mainz, Institut f\"ur Kernphysik, 55099 Mainz, Germany}}
\maketitle

\begin{abstract}
  We present an analysis of multiplicities of charged hadrons from 2016 data on pure proton lH$_2$ target.
  The data cover the following kinematic range : $Q^2$ > 1 (GeV/c)$^2$, 0.1 < $y$ < 0.7, 0.004 < $x$ < 0.4,
  0.2 < $z$ < 0.85.
\end{abstract}

\newpage

\section{Introduction}

In this note, we present the analysis of charged hadron multiplicities from data taken in 2016 on a pure proton lH$_2$ target.
The present release is prepared in view of the XXVII International Workshop on Deep-Inelastic Scattering and Related Subject
conference in April 2019.

\section{Analysis of raw multiplicities}

The analysis is performed on COMPASS data recorded in 2016 using a 160 GeV muon beam incident on a pure proton target (lH$_2$).
Five periods of the 2016 data are analyzed : P07 (slot 2), P08 (slot 2), P09 (slot 2), P10 (slot 2) and P11 (slot 2).

The PHAST version used for this analysis is X.XXX.

\subsection{Event and hadron selection}


\section{Conclusions}
Here you briefly summarize your findings.

%++++++++++++++++++++++++++++++++++++++++
% References section will be created automatically
% with inclusion of "thebibliography" environment
% as it shown below. See text starting with line
% \begin{thebibliography}{99}
% Note: with this approach it is YOUR responsibility to put them in order
% of appearance.

\begin{thebibliography}{99}

% \bibitem{melissinos}
% A.~C. Melissinos and J. Napolitano, \textit{Experiments in Modern Physics},
% (Academic Press, New York, 2003).

\end{thebibliography}


\end{document}
