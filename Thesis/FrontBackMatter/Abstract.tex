% Abstract

%\renewcommand{\abstractname}{Abstract} % Uncomment to change the name of the abstract

\pdfbookmark[1]{Abstract}{Abstract} % Bookmark name visible in a PDF viewer

\begingroup
\let\clearpage\relax
\let\cleardoublepage\relax
\let\cleardoublepage\relax

\chapter*{Abstract}

The spin structure of the nucleon is studied at the COMPASS experiment. This subject is of special interest since the surprising finding of the European muon collaboration (EMC) that the contribution from the quark spins to the nucleon spin is rather small. This finding started a series of experiments looking to measure various contribution of to the nucleon spin. The question of the polarization of the sea quark is an important topic in the hadronic physics, especially for the strange quark polarization.

In order to better describe the quark polarization, a precise knowledge of the quarks Fragmentation Functions (FFs) into hadrons, which describes the final state hadronisation of quark $q$ into hadron $h$, is mandatory. The FFs can be extracted from hadron multiplicities produced in Deep Inelastic Scattering (DIS). At the COMPASS experiment, a $160$ GeV/$c$ muon beam is scattered off a fixed pure proton target (lH$_2$).

This thesis presents the measurement of charged hadron, identified pion, kaon and proton multiplicities from DIS data collected in 2016. It also details the improvements of the analysis with the DJANGOH event generator, which is used to better describe the inclusive and semi-inclusive radiative corrections in DIS, which are then used as correction factors to the multiplicities, and the adaptation of DJANGOH to the COMPASS Monte-Carlo chain. The radiative correction factors to the multiplicities are computed in a three dimensional binning in $x$, $y$ and $z$. For the first time a solid determination of radiative effects dependent on the $z$ variable has been achieved.

The data cover a large kinematic range : $Q^2 > 1$ (GeV/$c$)$^2$, $y$ $\in$ [$0.1$,$0.7$], $x$ $\in$ [$0.004$,$0.4$], $W$ $\in$ [$5$,$17$] GeV/$c^2$ and $z$ $\in$ [$0.2$,$0.85$]. The charged hadron multiplicities are obtained in a $3$-dimensional ($x,y,z$) binning yielding a total of $540$ bins. Several corrections factors that are applied to the multiplicities are discussed : acceptance correction, vector meson corrections, RICH efficiency unfolding, electron contamination correction and radiative corrections. The multiplicities, which represent about $1800$ data points in total, provide an important input for global QCD fit of world data at NLO, aiming at FFs determination and complete the previous results for the charged hadron multiplicities using the data taken in $2006$ of muon scattering off an isoscalar target ($^6$LiD).

The quark FFs into kaons are particularly important as they are necessary to better constrain the strange quark polarization. The $K^+$ and $K^-$ multiplicities were used to extract the favoured $D^{K}_{fav}$, unfavoured $D^{K}_{unf}$ and strange $D^{K}_{s}$ quark FFs with a fit at LO. The result of the fit points out that there is only a weak sensitivity to the strange quark of these measurements. The fit gives too much contribution to the favoured and unfavoured fragmentation functions at the expense of the strange fragmentation function. This needs a more detailed investigation and also a fit in NLO perturbative QCD.

\newpage

\chapter*{Résumé}

Un des buts de la collaboration COMPASS est l'étude de la structure de spin du nucléon. Ce sujet présente un intérêt particulier depuis les découvertes surprenantes de la European muon collaboration (EMC) selon lesquelles la contribution des spins de quarks au spin de nucléons est plutôt faible. Cette découverte a lancé la recherche des diverses contributions au spin du nucléon. La question de la polarisation des quarks de la mer est un sujet capital en physique hadronique, en particulier pour la polarisation du quark étrange.
Afin de mieux contraindre la polarisation des quarks, une connaissance précise des fonctions de fragmentation (FFs), qui expriment l'hadronisation d'un quark $q$ en un hadron $h$ dans l'état final, est nécessaire. Les FFs peuvent être extraites depuis les multiplicités de hadrons produites en Diffusion Inélastique Profonde (DIS). Les données ont été prises à COMPASS avec un faisceau de muons de $160$ GeV/$c$ diffusant sur une cible de protons pure (lH$^2$).

La présente thèse présente les mesures des multiplicités de hadron chargé, pion, kaon et proton identifiés, faites à partir des données SIDIS collectées en $2016$. Elle détaille aussi les améliorations apportées au générateur d'événement DJANGOH dans le but d'améliorer la description des corrections radiative inclusive et semi-inclusive dans DIS qui sont ensuite utilisées comme facteurs de corrections aux multiplicités et l'adaptation de DJANGOH à la chaîne Monte-Carlo de COMPASS. Les facteurs de correction radiative aux multiplicités sont calculés selon un binning en trois dimensions en $x$, $y$ et $z$. Pour la première fois, une détermination claire des effets radiatifs en fonction de la variable $z$ a été effectuée.

Les données couvrent une large plage cinématique: $Q^2 > 1$ (GeV/$c$)$^2$, $y$ $\in$ [$0.1$,$0.7$], $x$ $\in$ [$0.004$,$0.4$], $W$ $\in$ [$5$,$17$] GeV/$c^2$ et $z$ $\in$ [$0.2$,$0.85$]. Les multiplicités de hadrons chargés sont obtenues dans un binning en $3$ dimensions ($x$,$y$,$z$), ce qui donne un total de $540$ bins. De multiples facteurs de correction des multiplicités sont abordés : la correction d'acceptance, la correction des mésons vecteurs, l'unfolding du RICH, la correction due à la contamination par les électrons et les corrections radiatives. Les multiplicités, qui représentent environ $1800$ points de données au total, fournissent une contribution importante au fit global QCD des données mondiales à NLO visant à déterminer les FFs et complètent les résultats précédents sur les multiplicités de hadrons chargés en utilisant les données prises en 2006 de diffusion de muons sur une cible isoscalaire ($^6$LiD).

Les FF de quark en kaons sont particulièrement recherchées car elles sont nécessaires pour mieux contraindre la polarisation du quark étrange. Les multiplicités $K^+$ et $K^-$ ont été utilisées pour extraire la fragmentation de quark favorisée $D^{K}_{fav}$, non favorisée $D^{K}_{unf}$ et étrange $D^{K}_{s}$ avec un fit à LO. Le résultat du fit montre qu'il y a une mauvaise sensibilité au quark étrange de ces mesures. Ce fit donne trop de poids aux fonctions de fragmentation favoured et unfavoured aux dépens de la fonction de fragmentation strange. Cela nécessite une enquête plus détaillée et un fit de QCD perturbative à NLO.


\newpage

\chapter*{Zusammenfassung}

Am COMPASS-Experiment wird die Spinstruktur des Nukleons untersucht. Diese ist von besonderen Interesse seit der überraschenden Entdeckung durch die European Muon Collaboration (EMC), dass nur ein kleiner Teil des Nukleonspins von den Spins der Quarks stammt. Mit dieser Entdeckung begann die Suche nach den verschiedenen Bestandteilen. Diese sind durch die Spins der Quarks und Gluonen sowie deren Bahndrehimpulsen gegeben, wobei der aktuelle wird von den Beitrag der Quarkspins etwa 30\% beträgt. Die Frage nach der Polarisation der Seequarks ist ein wichtiges Thema in der Hadronphysik, insbesondere für die Strangequarkpolarisation.

Um die Quarkpolarisation genauer bestimmt zu können, ist eine genaue Kenntnis der Quarkfragmentationsfunktionen (FFs) in Hadronen erforderlich, die die Hadronisierung von einen Quark $q$ in ein Hadron $h$ beschreiben. Die FFs können aus Hadronenmultiplizitäten extrahiert werden, die durch inklusive tiefinelastische Streuung (DIS) erzeugt wurden. Beim COMPASS-Experiment wird ein $160$ GeV/$c$ Myonstrahl an einem ruhenden reinen Protontarget (lH$_2$) gestreut.

In dieser Arbeit wird die Messung von Multiplizitäten geladener Hadronen, identifizierter Pionen, Kaonen und Protonen anhand der $2016$ gesammelten DIS-Daten vorgestellt. Außerdem werden die Verbesserungen der Analyse durch die Verwendung des DJANGOH-Ereignis-generators zur Bestimmung der inklusiven und semi-inklusiven Strahlungskorrekturen in DIS erläutert. Diese Korrekturfaktoren werden an die Multiplizitäten eingebracht. Die Anpassung von DJANGOH an die COMPASS Monte-Carlo Kette wird beschrieben. Die Strahlungskorrekturfaktoren für die Multiplizitäten werden in einer dreidimensionalen Einteilung von $x$,$y$ und $z$-Intervallen. Hiermit wurde zum ersten Mal eine Bestimmung der Strahlungseffekte in Abhängigkeit von der $z$-Variablen durchgeführt.

Die Daten decken einen großen kinematischen Bereich ab: $Q^2 > 1$ (GeV/$c$)$^2$, $y$ $\in$ [$0.1$,$0.7$], $x$ $\in$ [$0.004$,$0.4$], $W$ $\in$ [$5$,$17$] GeV/$c^2$ und $z$ $\in$ [$0.2$,$0.85$]. Die geladenen Hadronmultiplizitäten werden in eine dreidimensionalen ($x$,$y$,$z$)-Einteilung bestimmt, was insgesamt 540 Intervalle ergibt. Verschiedene Korrekturfaktoren, die auf die Multiplizitäten angewendet werden, werden diskutiert: Akzeptanzkorrektur, Vektormesonkorrektur, RICH-Akzeptanzentfaltung, Elektronkontaminationskorrektur und Strahlungskorrektur. Die Multiplizitäten, die insgesamt etwa 1800 Datenpunkte repräsentieren, liefern einen wichtigen Input für die globale NLO pQCD-Anpassung der Weltdaten, mit denen FFs bestimmt werden. Außerdem werden die vorherigen Ergebnisse der Myonstreuung an einem isoskalaren Target ($^6$LiD) aus dem Jahr $2006$ vervollständigt.

Besonders gefragt sind die Quark-FFs in Kaonen, um die Strangequarkpolarisation besser einschränken zu können. Die $K^+$-und-$K^-$ Multiplizitäten werden verwendet, um die  favouris-ierte $D^{K}_{fav}$, die nicht favourisierte $D^{K}_{unf}$ und die Strangequark $D^{K}_{s}$ Quark-FFs mit einer Anpassung in LO pQCD zu extrahieren. Das Ergebnis der Anpassung weist darauf hin, dass die Daten nicht sehr empfindlich auf den Strangequarkbeitrag sind. Die Anpassung führt zu einem überraschend
grossen Beitrag von favourisierter und nicht favourisierter fragmentation auf Kosten der Strangefragmentationsfunktion. Daher sind in Zukunft weitere Untersuchungen und eine Anpassung in NLO perturbative QCD nötig.

% One of the goals of the COMPASS collaboration is the study of the nucleon spin structure. The question of the polarization of the sea quark is a burning issue in the hadronic physics, especially for the strange quark polarization. In order to better constrain the quark polarization, a precise knowledge of the quark Fragmentation Functions (FFs) into hadrons, which describes the final state hadronisation of quark $q$ into hadron $h$, is mandatory. The FFs can be extracted from hadron multiplicities produced in Deep Inelastic Scattering (DIS). Data were taken at COMPASS using a 160 GeV/$c$ muon beam scattering off a pure proton target (lH$_2$). This thesis presents the measurement of charged hadron (pions, kaons and protons) multiplicities from DIS data collected in 2016. It also details the improvements of the analysis using the DJANGOH event generator to better describe the inclusive and semi-inclusive radiative corrections in DIS that are then used as correction factors to the multiplicities and the adaptation of DJANGOH to the COMPASS Monte-Carlo chain. The data cover a large kinematical range : $Q^2 > 1$ (GeV/$c$)$^2$, $y$ $\in$ [0.1,0.7], $x$ $\in$ [0.004,0.4], $W$ $\in$ [5,17] GeV/$c^2$ and $z$ $\in$ [0.2,0.85]. The multiplicities, which represent about 1800 data points in total, provide an important input for global QCD fit of world data at NLO, aiming at FFs determination. The quark FFs into kaons are particularly looked for as they are necessary to better constrain the strange quark polarization.

% \vspace{2cm}

% Un des buts de la collaboration COMPASS est l'étude de la structure de spin du nucléon. La question de la polarisation des quarks de la mer est un sujet capital en physique hadronique, en particulier pour la polarisation du quark étrange. En vue de mieux contraindre la polarisation des quarks, une connaissance précise des fonctions de fragmentation (FFs), qui expriment l'hadronisation d'un quark $q$ en un hadron $h$ dans l'état final, est nécessaire. Les FFs peuvent être extraites depuis les multiplicités de hadrons produites en Diffusion Inélastique Profonde (DIS). Les données ont été prises à COMPASS avec un faisceau de muons de 160 GeV/$c$ diffusant sur une cible de protons pure (lH$^2$). La présente thèse présente les mesures des multiplicités de hadrons chargés (pions, kaons et protons) faites à partir des données SIDIS collectées en 2016. Elle détaille aussi les améliorations apportées au générateur d'événement DJANGOH dans le but d'améliorer la description des corrections radiative inclusive et semi-inclusive qui sont ensuite utilisées comme facteurs de corrections aux multiplicités. Les données couvrent un large spectre cinématique : $Q^2 > 1$ (GeV/$c$)$^2$, $y$ $\in$ [0.1,0.7], $x$ $\in$ [0.004,0.4], $W$ $\in$ [5,17] GeV/$c^2$ et $z$ $\in$ [0.2,0.85]. Ces multiplicités, qui représentent un total d'environ 1800 points de données, apportent une contribution importante aux fit QCD globaux des données mondiales à NLO, visant à la détermination des FFs. Les FFs de quarks en kaons sont particulièrement attendues car elles pourront mieux contraindre la polarization du quark étrange.

% \newpage

% Eines der Ziele der COMPASS Collaboration ist die Untersuchung der Nukleonspinstruktur. Die Frage der Polarisation des Seequarks ist ein brennendes Thema in der Hadronenphysik, insbesondere für die strange Quarkpolarisation. Um die Quark-Polarisation besser einschränken zu können, ist eine genaue Kenntnis der Quark-Fragmentierungsfunktionen (FFs) in Hadronen erforderlich, die die endgültige Hadronisierung von Quark $q$ in Hadron $h$ darstellen. Die FFs können aus Hadronenmultiplizitäten extrahiert werden, die mit Semi-Inklusive Tiefinelastische Streuung (DIS) erzeugt wurden. Die Daten wurden bei COMPASS von einem 160 GeV/$c$ Myonstrahl aufgenommen, der von einem reinen Protonentarget (lH$_2$) gestreut wurde. In dieser Arbeit wird die Messung der Multiplizität geladener Hadronen (Pionen, Kaonen und Protonen) anhand der 2016 gesammelten SIDIS-Daten vorgestellt. Außerdem werden die Verbesserungen des DJANGOH-Ereignisgenerators erläutert, um die dann vorliegenden inklusiven und semi-inklusiven Strahlungskorrekturen in DIS besser zu beschreiben als Korrekturfaktoren für die Multiplizitäten verwendet. Die Daten decken einen großen kinematischen Bereich ab:  $Q^2 > 1$ (GeV/$c$)$^2$, $y$ $\in$ [0.1,0.7], $x$ $\in$ [0.004,0.4], $W$ $\in$ [5,17] GeV/$c^2$ et $z$ $\in$ [0.2,0.85]. Diese Multiplizitäten, die insgesamt etwa 1800 Datenpunkte repräsentieren, liefern einen wichtigen Input für die globale QCD-Anpassung von Weltdaten an NLO mit dem Ziel der FFs-Bestimmung. Die Quark-FFs in Kaonen werden besonders erwartet, da sie die seltsame Quark-Polarisation besser einschränken können.

% One of the goals of the COMPASS collaboration is the study of the nucleon spin structure. The quarks, which are the constituants of the proton, can be studied by scattering high energy muons on protons. These muons probes the inside of the proton via the exchange of a photon which interacts with quarks. The quarks that are ejected from the proton then form new observable particles called hadrons. This is the hadronization process. The number of particles by muon sent is called hadron multiplicity. These multiplicities being linked to the hadronization process, it is possible to model the latter via the Fragmentation Functions (FFs). In this thesis, the different hadron (pion, kaon and proton) multiplicities are extracted from the data collected by COMPASS in 2016. These multiplicities provide an important input for the determination of the FFs which are universal objects needed for the interpretation of several physical results.

% Un des buts de la collaboration COMPASS est l'étude de la structure interne du proton. Les quarks, qui sont les constituants du proton, peuvent être étudiés en envoyant des muons sur des protons. Ces muons vont sonder l'intérieur du proton via l'échange d'un photon virtuel qui va interagir avec des quarks. Les quarks éjectés du proton vont ensuite former de nouvelles particules observables, nommées hadrons. C'est le processus d'hadronisation. Le nombre de particules obtenues par muon envoyé est appelé multiplicité de hadrons. Ces multiplicités étant reliées au processus d'hadronisation, il est possible de le modéliser au travers des Fonctions de Fragmentations (FFs). Dans cette thèse, les différentes multiplicités de hadrons (pion, kaon et proton) sont extraites des données collectées par COMPASS en 2016. Ces multiplicités apportent une contribution importante pour la détermination des FFs qui sont des objets universels nécessaires à l'interprétation de nombreux résultats physiques.

\endgroup

\vfill
