% Abstract

%\renewcommand{\abstractname}{Abstract} % Uncomment to change the name of the abstract

\pdfbookmark[1]{Abstract}{Abstract} % Bookmark name visible in a PDF viewer

\begingroup
\let\clearpage\relax
\let\cleardoublepage\relax
\let\cleardoublepage\relax

\chapter*{Abstract}

One of the goals of the COMPASS collaboration is the study of the nucleon spin structure. The question of the polarization of the sea quark is a burning issue in the hadronic physics, especially for the strange quark polarization. In order to better constrain the quark polarization, a precise knowledge of the quark Fragmentation Functions (FFs) into hadrons, which describes the final state hadronisation of quark $q$ into hadron $h$, is mandatory. The FFs can be extracted from hadron multiplicities produced in Deep Inelastic Scattering (DIS). Data were taken at COMPASS using a 160 GeV/$c$ muon beam scattering off a pure proton target (lH$_2$). This thesis presents the measurement of charged hadron (pions, kaons and protons) multiplicities from DIS data collected in 2016. It also details the improvements of the analysis using the DJANGOH event generator to better describe the inclusive and semi-inclusive radiative corrections in DIS that are then used as correction factors to the multiplicities and the adaptation of DJANGOH to the COMPASS Monte-Carlo chain. The data cover a large kinematical range : $Q^2 > 1$ (GeV/$c$)$^2$, $y$ $\in$ [0.1,0.7], $x$ $\in$ [0.004,0.4], $W$ $\in$ [5,17] GeV and $z$ $\in$ [0.2,0.85]. The multiplicities, which represent about 1800 data points in total, provide an important input for global QCD fit of world data at NLO, aiming at FFs determination. The quark FFs into kaons are particularly looked for as they are necessary to better constrain the strange quark polarization.

\vspace{2cm}

Un des buts de la collaboration COMPASS est l'étude de la structure de spin du nucléon. La question de la polarisation des quarks de la mer est un sujet capital en physique hadronique, en particulier pour la polarisation du quark étrange. En vue de mieux contraindre la polarisation des quarks, une connaissance précise des fonctions de fragmentation (FFs), qui expriment l'hadronisation d'un quark $q$ en un hadron $h$ dans l'état final, est nécessaire. Les FFs peuvent être extraites depuis les multiplicités de hadrons produites en Diffusion Inélastique Profonde Semi-Inclusive (SIDIS). Les données ont été prises à COMPASS avec un faisceau de muons de 160 GeV/$c$ diffusant sur une cible de protons pure (lH$^2$). La présente thèse présente les mesures des multiplicités de hadrons chargés (pions, kaons et protons) faites à partir des données SIDIS collectées en 2016. Elle détaille aussi les améliorations apportées au générateur d'événement DJANGOH dans le but d'améliorer la description des corrections radiative inclusive et semi-inclusive qui sont ensuite utilisées comme facteurs de corrections aux multiplicités. Les données couvrent un large spectre cinématique : $Q^2 > 1$ (GeV/$c$)$^2$, $y$ $\in$ [0.1,0.7], $x$ $\in$ [0.004,0.4], $W$ $\in$ [5,17] GeV et $z$ $\in$ [0.2,0.85]. Ces multiplicités, qui représentent un total d'environ 1800 points de données, apportent une contribution importante aux fit QCD globaux des données mondiales à NLO, visant à la détermination des FFs. Les FFs de quarks en kaons sont particulièrement attendues car elles pourront mieux contraindre la polarization du quark étrange.

\newpage

Eines der Ziele der COMPASS Collaboration ist die Untersuchung der Nukleonspinstruktur. Die Frage der Polarisation des Seequarks ist ein brennendes Thema in der Hadronenphysik, insbesondere für die strange Quarkpolarisation. Um die Quark-Polarisation besser einschränken zu können, ist eine genaue Kenntnis der Quark-Fragmentierungsfunktionen (FFs) in Hadronen erforderlich, die die endgültige Hadronisierung von Quark $q$ in Hadron $h$ darstellen. Die FFs können aus Hadronenmultiplizitäten extrahiert werden, die mit Semi-Inklusive Tiefinelastische Streuung (SIDIS) erzeugt wurden. Die Daten wurden bei COMPASS von einem 160 GeV/$c$ Myonstrahl aufgenommen, der von einem reinen Protonentarget (lH$_2$) gestreut wurde. In dieser Arbeit wird die Messung der Multiplizität geladener Hadronen (Pionen, Kaonen und Protonen) anhand der 2016 gesammelten SIDIS-Daten vorgestellt. Außerdem werden die Verbesserungen des DJANGOH-Ereignisgenerators erläutert, um die dann vorliegenden inklusiven und semi-inklusiven Strahlungskorrekturen in DIS besser zu beschreiben als Korrekturfaktoren für die Multiplizitäten verwendet. Die Daten decken einen großen kinematischen Bereich ab:  $Q^2 > 1$ (GeV/$c$)$^2$, $y$ $\in$ [0.1,0.7], $x$ $\in$ [0.004,0.4], $W$ $\in$ [5,17] GeV et $z$ $\in$ [0.2,0.85]. Diese Multiplizitäten, die insgesamt etwa 1800 Datenpunkte repräsentieren, liefern einen wichtigen Input für die globale QCD-Anpassung von Weltdaten an NLO mit dem Ziel der FFs-Bestimmung. Die Quark-FFs in Kaonen werden besonders erwartet, da sie die seltsame Quark-Polarisation besser einschränken können.

% One of the goals of the COMPASS collaboration is the study of the nucleon spin structure. The quarks, which are the constituants of the proton, can be studied by scattering high energy muons on protons. These muons probes the inside of the proton via the exchange of a photon which interacts with quarks. The quarks that are ejected from the proton then form new observable particles called hadrons. This is the hadronization process. The number of particles by muon sent is called hadron multiplicity. These multiplicities being linked to the hadronization process, it is possible to model the latter via the Fragmentation Functions (FFs). In this thesis, the different hadron (pion, kaon and proton) multiplicities are extracted from the data collected by COMPASS in 2016. These multiplicities provide an important input for the determination of the FFs which are universal objects needed for the interpretation of several physical results.

% Un des buts de la collaboration COMPASS est l'étude de la structure interne du proton. Les quarks, qui sont les constituants du proton, peuvent être étudiés en envoyant des muons sur des protons. Ces muons vont sonder l'intérieur du proton via l'échange d'un photon virtuel qui va interagir avec des quarks. Les quarks éjectés du proton vont ensuite former de nouvelles particules observables, nommées hadrons. C'est le processus d'hadronisation. Le nombre de particules obtenues par muon envoyé est appelé multiplicité de hadrons. Ces multiplicités étant reliées au processus d'hadronisation, il est possible de le modéliser au travers des Fonctions de Fragmentations (FFs). Dans cette thèse, les différentes multiplicités de hadrons (pion, kaon et proton) sont extraites des données collectées par COMPASS en 2016. Ces multiplicités apportent une contribution importante pour la détermination des FFs qui sont des objets universels nécessaires à l'interprétation de nombreux résultats physiques.

\endgroup

\vfill
