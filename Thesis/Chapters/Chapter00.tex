% Chapter 0

\chapter*{Introduction} % Chapter title

\label{ch:intro} % For referencing the chapter elsewhere, use \autoref{ch:name}

%----------------------------------------------------------------------------------------

The internal structure of the nucleon was first observed at the Standford Linear Accelerator Center (SLAC) in 1969 \cite{SLAC} by scattering high energy electron on nucleons. In this process, electrons interacted with internal components of the nucleon with an exchange of a virtual photon $\gamma^*$. Some years before, theories about substructure of hadrons were developed by Gell-Mann (\textit{quarks}) \cite{Gellman} and Zweig (\textit{aces}) \cite{Zweig} describing the nucleon as a system composed by three point-like objects. In response to the aforementionned SLAC results, Feynman developed the parton model \cite{PartonModel}, a model analoguous to the Gell-Mann's quark model. Fusing both models together, the \textit{Quark Parton Model} (QPM) was born, describing the nucleon as being composed by three \textit{valence quarks} and by quark-antiquark pairs (\textit{sea quarks}). Following SLAC discovery, other experiments tried to do lepton-nucleon scattering but at higher energies and with different types of leptons (e.g. muons or neutrinos). They found that in addition to quarks there should be other components of the nucleon. At the same time, the \textit{Quantum ChromoDynamics} (QCD) was developped and introduced gluons as the vector bosons of the strong interaction but also the concept of confinement, which explains why one cannot observe isolated quarks.

This lepton-nucleon scattering, named \textit{Deep Inelastic Scattering} (DIS), has a final state composed by the scattered lepton and a hadronic system. If both the scattered lepton and at least one hadron are detected in the final state, corresponding to a \textit{Semi-Inclusive} measurement (SIDIS), the corresponding cross-section is the one of a hard scattering process (lepton-quark scattering in \textit{perturbative} QCD (pQCD)) convoluted with two probability densities : the \textit{Parton Distribution Functions} (PDFs) and the \textit{Fragmentation Functions} (FFs). The PDFs are parametrizing the partonic structure of, in our case, the nucleons, while the FFs are parametrizing the hadronisation process, which is the formation of hadrons out of quarks scattered by the process. These quantities are expected to be universal in the sense that they are process independent : they do not depend on the measured process. The PDFs of $u$, $\bar{u}$, $d$ and $\bar{d}$ quarks are currently well described with high precision measurement from experiment, while the ones of $s$ and $\bar{s}$ are still not well constrainted. They are affected by large incertainties coming notably from the poor knowledge of strange quark into hadron FFs. They are measured from three different processes : electron-positron annihilation, SIDIS and hadron-hadron collisions. The FFs are useful in the determination of the spin structure functions through SIDIS. While they are well known for the first generation of quarks, they are still not well determined for higher mass quark with a discrepancy up to a factor $3$ between parametrizations for the strange quark. They thus constitute the largest uncertainty for the determination of the strange quark polarization in SIDIS.

One of the goals of the COMPASS experiment at CERN is to study the nucleon spin structure. To reach it SIDIS data were taken using a $160$ GeV $\mu^+/\mu^-$ beam and a pure proton ($lH_2$) target, adding up to the data already taken using a polarized $160$ GeV muon beam and a polarized target ($^6LiD$ or $NH_3$). The knowledge of the quark FFs is crucial in the determination of quark polarization to achieve full flavour separation, as strange quark FFs are again the main source of uncertainties \cite{COMPASSstrange}. COMPASS has already been bringing measurements with an isoscalar $^6LiD$ target to the world data to better constrain FFs through world data QCD fit. In order to continue to contribute with new measurements, new extraction of charged hadrons multiplicities (averaged number of hadron produced per DIS events) from COMPASS data with a pure proton target was decided.

The work done in this thesis is about the entire analysis chain of extraction of charged unidentified and identified hadrons (pion, kaon, proton) multiplicities.

The first part will go through the theoretical framework used for the analysis. The kinematic variables used to describe the DIS and SIDIS processes and their corresponding cross-sections are presented. The QPM and its QCD improved version is discussed. The PDFs and FFs are introduced and a state of the art picture of the current knowledge of the FFs is drawn.

The second part describes the COMPASS experiment. The main components of the COMPASS apparatus are shortly described while the \textit{Ring Imaging CHerenkov} (RICH) detector is more thourougly adressed as one of the main detector used in the analysis. The determination of RICH performances concerning particle identification and misidentification are also discussed.

The third part is about the DJANGOH Monte-Carlo generator, which allows generation of radiative events within our Monte-Carlo simulation and more broadly radiative corrections estimation.

The fourth and final part is focused on the extraction of charged unidentified ($h^+/h^-$) and identified ($\pi^+/\pi^-,K^+/K^-,p/\bar{p}$) multiplicities from COMPASS data as well as leading order extraction of FFs from the aforementionned kaon multiplicities.
