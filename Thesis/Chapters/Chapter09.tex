% Chapter 9

\chapter{Analysis of 2016 raw multiplicities} % Chapter title

\label{ch:raw} % For referencing the chapter elsewhere, use \autoref{ch:name}

The 2006 SIDIS COMPASS hadron multiplicities results, based on data taken with an isoscalar target ($^6$LiD),
do not constrain fermly the strange quark fragmentation function.
With the analysis of new data taken on pure proton target (LH$_2$), the results will provide
a new set of equations linking the multiplicities with the fragmentation functions but still
involving the same quark fragmentation functions we are interested in. The fact that the proton
and deuteron data can be fitted together will add constrains to the fragmentation function extraction.
In order to perform this kind of study, one need a precision of 5 to 7\% on the multiplicities.

The analysis is performed on COMPASS data recorded in 2016 using a 160 GeV muon beam incident on a proton
target (LH$_2$). Five weeks of the 2016 data are analyzed.

%----------------------------------------------------------------------------------------

\section{Method of extraction}

The method of extraction of the multiplicities follow several steps. For each selection steps,
a list of cuts is applied on both geometrical and physical quantities. First the DIS events are selected
and then the SIDIS events (hadrons) are selected. For the DIS event selection, a study on the target radius
had to be made in order to determine the optimal value for the target cut. When the event selection is done,
the hadrons have to be identified between pions, kaons and protons and the identification has to be corrected
according to the RICH detection efficiency and purity in a process called unfolding. The obtained raw multiplicities are
then binned and corrected with the detector acceptance. The general event reconstruction codes for COMPASS are
used. A personal analysis code is developed to study the SIDIS channel and select pion or kaon production events.

In the following, the number of residual events after each cut will be given between parentheses next to each cut.
When the number is not specified, it can mean that the cut was done in a pre-analysis or that the recovery of the
number is too complex due to the transversal implementation of the cut in the code (e.g. for the $\nu$ cut).

The analysis is performed on COMPASS data recorded in 2016 using a 160 GeV muon beam incident on a pure proton target (lH$_2$).
Five periods of the 2016 data are analyzed : P07, P08, P09, P10 and P11.

%------------------------------------------------

\section{DIS event selection}

For the DIS event selection ($\mu p \rightarrow \mu X$), the following list of cuts is applied :
\begin{enumerate}
  \item Events with Best Primary Vertex (53.8 M events, 100\%)
  \item Events with reconstruted scattered muon (53.8 M events, 100\%)
  \item Events with primary interaction in the target material, target radius cut (explained in Section \ref{}, 22.7 M events, 42.1\%)
  \item Events with energy of beam muon energy in range [140 GeV, 180 GeV] (22.7 M events, 42.1\%)
  \item Events with a well reconstructed beam track (so-called 'BMS cut') (20.8 M events, 38.6\%)
  \item Events with $\chi^2$/ndf $<$ 10 for a well reconstructed beam track (20.8 M events, 38.6\%)
  \item Events with muon beam trajectory crossing entirely the target cell (20.0 M events, 37.2\%)
  \item Events with $\chi^2$/ndf $<$ 10 for a well reconstructed scattered muon track (20.0 M events, 37.2\%)
  \item Events with Z coordinate of the first measured hit of scattered muon $<$ 350 cm ($Z_{SM1}$) (19.9 M events, 37.1\%)
  \item Events with Middle, Ladder, Outer or LAST trigger (19.9 M events, 37.1\%)
  \item Events with $Q^2>1$ (GeV/c)$^2$ (DIS validity, 13.9 M events, 25.8\%)
  \item Events with $0.1 < y < 0.9$ (6.3 M events, 11.7\%)
  \item Events with $5 < W < 17$ GeV/c$^2$ (6.3 M events, 11.6\%)
  \item Events with $0.004 < x < 0.4$ (6.3 M events, 11.6\%)
  \item Events with $\nu$ cut
\end{enumerate}

The cut on the kinematic variable $\nu$ was implemented to reject events that contain hadrons outside of the measured
momentum range of 12 - 40 GeV/c. The criteria is defined by :
\begin{equation}
  \nu_{max} = \frac{\sqrt{(p^2_{max}+m^2_h)}}{z_{max}}
\end{equation}
\begin{equation}
  \nu_{min} = \frac{\sqrt{(p^2_{min}+m^2_h)}}{z_{min}}
\end{equation}

where $p_{max}$ (resp. $p_{min}$) is the hadron momentum limit of 40 GeV/c (resp. 12 GeV/c), $z_{max}$ (resp. $z_{min}$)
is the upper (resp. lower) value of the $z$-bin and $m_h$ is the mass of the considered hadron.

%------------------------------------------------

\section{Target cut evaluation}

TBA

%------------------------------------------------

\section{Hadron selection}

For the hadron selection ($\mu p \rightarrow \mu hX$), the following list of cuts is applied :
\begin{enumerate}
  \item Particle is not a scattered muon (15.2 M hadrons, 100\%)
  \item Maximum radiation length cumulated along all the trajectory < 15 radiation lengths (14.9 M hadrons, 98.3\%)
  \item $\chi^2$/ndf $<$ 10 for the hadron track (14.6 M hadrons, 96.1\%)
  \item Z coordinate of the first measured hit < 350 cm (14.6 M hadrons, 95.9\%)
  \item Z coordinate of the last measured hit > 350 cm (14.6 M hadrons, 93.6\%)
  \item $0.01 < \theta_{RICH} < 0.12$ (at RICH entrance, 9.5 M hadrons, 62.3\%)
  \item $x^2_{RICH} + y^2_{RICH} > 25$ cm$^2$ (rejection of RICH pipe) (9.3 M hadrons, 61.5\%)
  \item $12 < p_h < 40$ GeV/c (2.5 M hadrons, 16.5\%)
  \item $0.2 < z < 0.85$ (1.9 M hadrons, 13.1\%)
\end{enumerate}

%----------------------------------------------------------------------------------------

\section{Particle Identification with RICH detector}

The $\pi$ and $K$ particle identification (PID) is performed by the RICH detector.

The method used for the RICH particle identification is described in \ref{}. The idea is the following : when a particle
is detected, six likelihood functions are calculated ($\pi$, $K$, $p$, $e$, $\mu$ and the background) and are then
compared to make the particle identification. The evaluation is done separately for pions, kaons and protons. The largest
value corresponds to the maximal probability. The method is improved by looking further to $LH(2^{nd})$ which is the second
highest value of the four compared likelihood values ($\pi$, $K$, $p$ and the background). The electron and muon likelihood
are not considered in the assignment of $LH(2^{nd})$ as in the chosen momentum range (2 to 40 GeV/c) the RICH detector can
not be used to efficiently distinguish electrons from $\pi$.

All $\pi$, $K$ and $p$ probabilities are needed for the unfolding.

\begin{enumerate}
  \item Pion selection
  \begin{itemize}
    \item $LH(\pi) > 0$
    \item $LH(\pi) > LH(K)$, $LH(p)$ and $LH(bgd)$. In case $LH(e) > 1.8LH(\pi)$, one must consider the electron hypothesis
    in the previous comparison.
    \item $\frac{LH(\pi)}{LH(2^{nd})}>1.02$
    \item $\frac{LH(\pi)}{LH(bgd)}>2.02$
  \end{itemize}
  \item Kaon selection
  \begin{itemize}
    \item $LH(K) > 0$
    \item $LH(K) > LH(\pi)$, $LH(p)$ and $LH(bgd)$. In case $LH(e) > 1.8LH(\pi)$, one must consider the electron hypothesis
    in the previous comparison.
    \item $\frac{LH(K)}{LH(2^{nd})}>1.08$
    \item $\frac{LH(K)}{LH(bgd)}>2.08$
  \end{itemize}
  \item Proton selection
  Three cases are considered depending on the momentum $p$ of the particle and are defined but the kaon threshold ($\simeq 8.9$ GeV/c)
  and proton threshold ($\simeq 17.95$ GeV/c)
  \begin{enumerate}[(a)]
    \item Kaon threshold $< p \leq$ proton threshold - 5 GeV/c
    \item $p >$ proton threshold + 5 GeV/c
    \begin{itemize}
      \item $LH(p) > 0$
      \item $LH(p) > LH(\pi)$, $LH(K)$ and $LH(bgd)$. In case $LH(e) > 1.8LH(\pi)$, one must consider the electron hypothesis
      in the previous comparison.
      \item $\frac{LH(p)}{LH(2^{nd})}>1$
    \end{itemize}
    \item Proton threshold - 5 GeV/c $< p <$ proton threshold + 5 GeV/c
    \begin{itemize}
      \item Using (a) and (b) simultaneously.
    \end{itemize}
  \end{enumerate}
\end{enumerate}

%----------------------------------------------------------------------------------------

\section{RICH unfolding based on efficiency matrices}

The performance of the RICH is not perfect : in terms of efficiency and purity, some particles
are misidentified.

The unfolding procedure is needed to correct the yield of identified hadrons for RICH detection efficiency.
In order to perform this correction, the RICH actual performance is evaluated from real data. The result of
this evaluation is presented through RICH efficiency probabilities matrices, $M_{RICH}$, binned in momentum
and angle :

\begin{itemize}
  \item $p$ {3,12,13,15,17,19,22,25,27,30,35,40} GeV/c
  \item $\theta$ {0.01,0.04,0.12} rad
\end{itemize}

The 3-by-3 matrices $M_{RICH}$ give a relation between the vector of true hadron $T_h$ and the vector of
identified hadron $I_h$

\begin{equation}
\begin{bmatrix}
I_{\pi} \\
I_K \\
I_p
\end{bmatrix}
=
\begin{bmatrix}
\epsilon(\pi \rightarrow \pi) & \epsilon(K \rightarrow \pi) & \epsilon(p \rightarrow \pi)\\
\epsilon(\pi \rightarrow K) & \epsilon(K \rightarrow K) & \epsilon(p \rightarrow K) \\
\epsilon(\pi \rightarrow p) & \epsilon(K \rightarrow p) & \epsilon(p \rightarrow p)
\end{bmatrix}
\begin{bmatrix}
T_{\pi} \\
T_K \\
T_p
\end{bmatrix}
\end{equation}

The coefficients of the $M_{RICH}$, $\epsilon{t \rightarrow i}$, are the probabilities that a true hadron
$t$ is identified as a hadron of type $i$. These probabilities have been determined as described in \ref{}.

By performing a matrix inversion, one can obtain the unfolded number of hadrons with the Eq.\ref{} :

\begin{equation}
  \overrightarrow{T_h} = M^{-1}_{RICH}\overrightarrow{I_h}
\end{equation}

where $M^{-1}_{RICH}$ coefficients are weights with which each identified hadron is counted as a pion, kaon
or proton.

\begin{table}
  \caption{}
  \label{}
  %Table of hadron counting
\end{table}

%----------------------------------------------------------------------------------------

\section{Errors on the RICH unfolding}

The RICH matrices are built using the statistical errors associated to the original probability matrix.

\begin{equation}
M^{\pm}_{RICH}
=
\begin{bmatrix}
\epsilon(\pi \rightarrow \pi)\pm\sigma_{\epsilon(\pi \rightarrow \pi)} & \epsilon(K \rightarrow \pi)\pm\sigma_{\epsilon(K \rightarrow \pi)} & \epsilon(p \rightarrow \pi)\pm\sigma_{\epsilon(p \rightarrow \pi)}\\
\epsilon(\pi \rightarrow K)\pm\sigma_{\epsilon(\pi \rightarrow K)} & \epsilon(K \rightarrow K)\pm\sigma_{\epsilon(K \rightarrow K)} & \epsilon(p \rightarrow K)\pm\sigma_{\epsilon(p \rightarrow K)} \\
\epsilon(\pi \rightarrow p)\pm\sigma_{\epsilon(\pi \rightarrow p)} & \epsilon(K \rightarrow p)\pm\sigma_{\epsilon(K \rightarrow p)} & \epsilon(p \rightarrow p)\pm\sigma_{\epsilon(p \rightarrow p)}
\end{bmatrix}
\end{equation}

As an inversion of the probabilities matrices is done in the analysis, the propagation of errors through the
inversion operation has to be performed. A calculation of the uncertainties on the probabilities yields \cite{} :

\begin{equation}
  [\sigma^{-1}_i]^2 = \epsilon^{-1}_{ip}\epsilon^{-1}_{ir}cov(\epsilon_{pq},\epsilon_{rs})\epsilon^{-1}_{qi}\epsilon^{-1}_{si} + [\epsilon^{-1}_{ik}\sigma_k]^2
\end{equation}

This equation leads to the inverse error matrices with associated statistical errors :

\begin{equation}
[M^{\pm}_{RICH}]^{-1}
=
\begin{bmatrix}
\epsilon^{-1}(\pi \rightarrow \pi)\pm\sigma^{-1}_{\epsilon(\pi \rightarrow \pi)} & \epsilon^{-1}(K \rightarrow \pi)\pm\sigma^{-1}_{\epsilon(K \rightarrow \pi)} & \epsilon^{-1}(p \rightarrow \pi)\pm\sigma^{-1}_{\epsilon(p \rightarrow \pi)}\\
\epsilon^{-1}(\pi \rightarrow K)\pm\sigma^{-1}_{\epsilon(\pi \rightarrow K)} & \epsilon^{-1}(K \rightarrow K)\pm\sigma^{-1}_{\epsilon(K \rightarrow K)} & \epsilon^{-1}(p \rightarrow K)\pm\sigma^{-1}_{\epsilon(p \rightarrow K)} \\
\epsilon^{-1}(\pi \rightarrow p)\pm\sigma^{-1}_{\epsilon(\pi \rightarrow p)} & \epsilon^{-1}(K \rightarrow p)\pm\sigma^{-1}_{\epsilon(K \rightarrow p)} & \epsilon^{-1}(p \rightarrow p)\pm\sigma^{-1}_{\epsilon(p \rightarrow p)}
\end{bmatrix}
\end{equation}

%----------------------------------------------------------------------------------------

\section{Kinematic binning}

The raw multiplicities are evaluated in bins of the Bjorken variable $x$, the muon energy fraction carried
by the virtual photon $y$ and the virtual photon energy fraction carried by final state hadron $z$. They
are calculated with the following formula :

\begin{equation}
  \frac{dM^h(x,y,z)}{dz}=\frac{1}{N^{DIS}_{Events}(x,y)}\frac{dN^{DIS}_{h}(x,y,z)}{dz}
\end{equation}

where $N^{DIS}_{Events}$ is the number of DIS events and $N^{DIS}_{h}$ is the number of
hadrons after RICH unfolding. As in practise, the multiplicities are measured in bins of
x (9 bins), y (6 bins) and z (12 bins), the calculated multiplicities can be expressed as :

\begin{equation}
  M^h_{raw}(x,y,z) = \frac{N^{DIS}_{h}(x,y,z)/\delta z}{N^{DIS}_{Events}}
\end{equation}

where $\delta z$ is the width of the z bin. For the multiplicities extraction, the binning in
$x$, $y$ and $z$ is the following :

\begin{itemize}
  \item $x$ {0.004,0.01,0.02,0.03,0.04,0.06,0.1,0.14,0.18,0.4}
  \item $y$ {0.1,0.15,0.2,0.3,0.5,0.7,0.9}
  \item $z$ {0.2,0.25,0.3,0.35,0.4,0.45,0.5,0.55,0.6,0.65,0.7,0.75,0.85}
\end{itemize}

%----------------------------------------------------------------------------------------

\section{Detector acceptance}

The COMPASS detector does not cover the full phase-space then the measured multiplicities have
to be corrected for the finite detector acceptance of the order of 70\%. The correction is
done using a Monte-Carlo dataset containing about 400 million events generated in the kinematic
region $Q^2 > 0.8$ (GeV/c)$^2$, x $\in$ [10$^{-4}$], y $\in$ [0.05,0.95].

The events are created with DJANGOH generator with parametrization of the parton distribution functions
(MSTW08). In addition, the use of JETSET inside DJANGOH allows the hadronization of quarks q to final-state
hadrons h according to the Lund model. The COMPASS high $p_T$ tuning was used, resulting in a good description
of real data as shown for DIS and hadrons in Fig.\ref{}.

The same DIS event and unidentified hadron selection that are used on real data (except the BMS cut) are applied
to the MC data sample for reconstructed MC events and particles.

The data are processed through a GEANT4 model of the spectrometer, TGEANT, and events are reconstructed with the
same CORAL version as for the real data.

The acceptance involves both reconstructed and generated particles. In both cases, the particle ID is taken from
the MC truth. The following selection is made on the generated events and particles :

\begin{enumerate}
  \item Energy of the beam muon in range [140,180] GeV
  \item Z coordinate of event vertex ($z_{vtx}$) within the target region
  \item $Q^2>1$ (GeV/c)$^2$
  \item $0.1 < y < 0.9$
  \item $0.004 < x < 0.4$
  \item $5 < W < 17$ GeV/c$^2$
  \item $\nu$ range used in data
  \item $0.2 < z < 0.85$
\end{enumerate}

In the following, $r$ and $g$ refers to 'reconstructed' and 'generated' quantities.

The acceptance is determined as the ratio of reconstructed multiplicities $M^h_r$ over the generated multiplicities $M^h_g$
and is binned in $x$, $y$ and $z$ :

\begin{equation}
  A^h(x,y,z) = \frac{M^h_r(x,y,z)}{M^h_g(x,y,z)}=\frac{N^h_r(x,y,z)/N^{DIS}_r(x,y,z)}{N^h_g(x,y,z)/N^{DIS}_g(x,y,z)}
\end{equation}

where $x_g$, $y_g$ and $z_g$ are the generated kinematic values and $x_r$, $y_r$ and $z_r$ are the reconstructed kinematic
values. Used in this fashion, the kinematic bin smearing due to reconstruction limitations is accounted for. A more rigorous
bin smearing correction would involve an unfolding procedure but is not done in this analysis.

For this method, the error estimation is difficult to rigorously calculate as the numbers of evaluated hadrons and DIS events,
in both the reconstructed and generated case, are not independent. An estimation is made by considering that the hadrons numbers
and DIS events are independent of each other.

Due to the $z$ kinematic bin migration effects, there exist particles in $N_r$ which are independent from $N_g$. Decomposing $N_r$
into two independent samples namely $N_{r^0}$ which are contained in $N_g$ and $N_{r'}$ which are not, the final acceptance error yields :

\begin{equation}
  \begin{split}
    E^2_{acc} = \left (\frac{G_D}{R_D+R'_{D}}\right )^2\left [\frac{(R_h+A)(G_h-R_h+1)}{(G_h+2)^2(G_D+3)}+\frac{R'_{h}}{G^2_h}+\frac{R'^2_h}{G^3_h}\right ] \\
                + \left (\frac{G_D}{R_D+R'_{D}}\right )^4\left (\frac{R_h+R'_h}{G_h}\right )^2\left [\frac{(R_D+1)(G_D-R_D+1)}{(G_D+2)^2(G_D+3)}+\frac{R'_D}{G^2_D}+\frac{R'^2_D}{G^3_D}\right ]
  \end{split}
\end{equation}

where $G_h$ (resp. $G_D$) are the generated hadrons (resp. DIS events) in a given $x$, $y$, $z$ bin, $R_h$ (resp. $R_D$) the reconstructed
hadrons (resp. DIS events) and $R'_h$ (resp. $R'_D$) all other particles (resp. events) that are reconstructed as hadrons (resp. DIS events)
in a given $x$, $y$, $z$ bin.

The correction is then applied to the raw multiplcities :

\begin{equation}
  M^h(x,y,z) = \frac{M^h_{raw}(x,y,z)}{A^h(x,y,z)}
\end{equation}

%----------------------------------------------------------------------------------------

\section{Diffractive vector meson correction}

It is usually assumed that hadrons produced in SIDIS originate from lepton-parton scattering. Nevertheless the scattering of a lepton
off a nucleon can also result in the diffractive production of vector mesons. These particles decay into lighter mesons that cannot be
distinguished from the one resulting from the hadronization of a quark originating from the target nucleon. This implies that fragmentation
functions extracted from multiplicities contaminated with diffractive vector mesons would violate universality, as they would be process
dependent. However, this is a complex theoretical discussion so the multiplicities both with and without subtracting the diffractive vector
meson contribution are calculated as well as the separate correction factors for DIS events and hadrons.

For kaons, the dominant vector meson contribution comes from the diffractive production of $\rho^0$ and $\Phi$ :
\begin{equation}
    \gamma * p \rightarrow \rho^0 p \rightarrow p\pi^+\pi^-
    \gamma * p \rightarrow \Phi p \rightarrow pK^+K^-
\end{equation}

This process is mainly exclusive but in 20\% of cases a diffractive dissociation of the target nucleon occurs. Other channels (excited $\rho$, $\omega$, etc.)
are expected to contribute much less and are not taken into account. As pions and kaons stemming from diffractive
vector meson decay cannot be separated from the one resulting from SIDIS, the evaluation of their contribution to the multiplicities is based on a
Monte Carlo study. Three Monte Carlo samples are produced based on different generators (SIDIS using DJANGOH, diffractive $\Phi$ using HEPGEN++) and
the same event reconstruction chain. For the diffractive vector meson samples, both exclusive events and events with diffractive dissociation of the
proton are simulated. The $\rho^0$ sample includes nuclear effects (coherent production and nuclear absorption).

The fraction of pions (resp. kaons) resulting from a diffractive $rho^0$ (resp. $\Phi$) is calculated in the same binning as the raw multiplicities as :

\begin{equation}
  \begin{split}
    f^{\pi}_{\rho^0}(x,y,z) = \frac{N^{\pi}_{HEPGEN++}(x,y,z)}{N^{\pi}_{DJANGOH}(x,y,z)+N^{\pi}_{HEPGEN++}(x,y,z)} \\
    f^K_{\Phi}(x,y,z) = \frac{N^K_{HEPGEN++}(x,y,z)}{N^K_{DJANGOH}(x,y,z)+N^K_{HEPGEN++}(x,y,z)}
  \end{split}
\end{equation}

where $N^{\pi}_{HEPGEN++}$, $N^{\pi}_{DJANGOH}$, $N^K_{HEPGEN++}$ and $N^K_{DJANGOH}$ are the number of kaons reconstructed from the HEPGEN++ and DJANGOH MC samples normalized by the corresponding
MC luminosity ($L_{MC}$). The luminosity depends on the event weighting and the process cross-section $\sigma_{int}$ (DIS for DJANGOH event and diffractive
vector meson production for HEPGEN++ events). The final weighted number of kaons is summarized in Table \ref{}.

\begin{equation}
  \sum_{events} w_i = L_{MC} \cdot \sigma{int}
\end{equation}

\begin{table}
  \caption{}
  \label{}

\end{table}

The diffractive vector meson events can also lead to a contamination in DIS events. Here, the two channels studied are diffractive $\rho^0$ and $\Phi$
with the fraction of the contamination expressed in Eqs. \ref{}. Contrary to previous Eq. \ref{}, the denominator only includes the DIS events from the
DJANGOH generator because the cross-section used to generate the DJANGOH sample takes into account the diffractive contribution.

\begin{equation}
  \begin{split}
    f^{\rho^0}_{DIS}(x,y,z) = \frac{N^{DIS}_{\rho^0,HEPGEN++}(x,y,z)}{N^{DIS}_{DJANGOH}(x,y,z)} \\
    f^{\Phi}_{DIS}(x,y,z) = \frac{N^{DIS}_{\Phi,HEPGEN++}(x,y,z)}{N^{DIS}_{DJANGOH}(x,y,z)}
  \end{split}
\end{equation}

The total contribution from the diffractive vector-meson contribution ($f^{VM}_{DIS}$) to the DIS sample is the sum of the $f^{\rho^0}_{DIS}$ and $f^{\Phi}_{DIS}$.
The final correction reads as follows :

\begin{equation}
  \begin{split}
  B^h(x,y,z) = \frac{ \frac{N^{\pi}(x,y,z)}{N^h(x,y,z)}\left (1-f^{\pi}_{\rho^0}(x,y,z)\right )
                   + \frac{N^K(x,y,z)}{N^h(x,y,z)}\left (1-f^{K}_{\Phi}(x,y,z)\right ) + \frac{N^p(x,y,z)}{N^h(x,y,z)} }{1-f^{VM}_{DIS}(x,y,z)} \\
  B^{\pi}(x,y,z) = \frac{1-f^{\pi}_{\rho^0}(x,y,z)}{1-f^{VM}_{DIS}(x,y,z)} \\
  B^K(x,y,z) = \frac{1-f^{K}_{\Phi}(x,y,z)}{1-f^{VM}_{DIS}(x,y,z)}
  \end{split}
\end{equation}

%----------------------------------------------------------------------------------------

\section{Error associated to RICH unfolding}

The first stage of pion identification is based on the likelihood ratios : $LH(\pi)/LH(2^{nd})$ and $LH(\pi)/LH(bgd)$. These cuts are optimized to minimize the
pions misidentified as kaons. The systematic error associated to the selection of these cuts is performed varying the cuts around optimized values. Two sets of
cuts \textit{loose} and \textit{severe} were used.

\begin{table}
  \caption{}
  \label{}

\end{table}

To evaluate the systematic error associated to the selection of the particle likelihood cuts, the particle identification is performed using the \textit{loose}
and \textit{severe} sets of likelihood cuts and the corresponding RICH probability matrices and final multiplicities are extracted ($M^{h^{\pm},loose}_{raw}$
and $M^{h^{\pm},severe}_{raw}$ respectively). The largest difference between $M^{h^{\pm},loose}_{raw}$ and $M^{h^{\pm},severe}_{raw}$ with the nominal
multiplicity $M^{h^{\pm}}_{raw}$ is taken as an estimate of the systematic error :

\begin{equation}
  \sigma^{RICH_LH}_{sys} = MAX(|M^{h^{\pm},loose}_{raw}-M^{h^{\pm}}_{raw}|,|M^{h^{\pm},severe}_{raw}-M^{h^{\pm}}_{raw}|)
\end{equation}

The difference between the altered RICH probability matrices and the optimal one are plotted in Fig.\ref{}. For pions and protons, the largest differences
(<5\%) are observed in the high momentum $p_h$ region. For kaons the difference reaches 4\% at low $p_h$ ; small differences (<1\%) are observed at the
highest $p_h$ value.

A second source of systematic error is that associated with the calculation of the RICH probability matrices. This is estimated by generating two sets of altered
RICH probability matrices. As represented in Eq.\ref{} the matrices are constructe using the statistical error associated to the original probability matrix elements.

\begin{equation}
  M^{\pm}_{RICH}
  =
  \begin{bmatrix}
  P(\pi \rightarrow \pi)\pm\sigma_{P(\pi \rightarrow \pi)} & P(K \rightarrow \pi)\mp\sigma_{P(K \rightarrow \pi)} & P(p \rightarrow \pi)\mp\sigma_{P(p \rightarrow \pi)}\\
  P(\pi \rightarrow K)\mp\sigma_{P(\pi \rightarrow K)} & P(K \rightarrow K)\pm\sigma_{P(K \rightarrow K)} & P(p \rightarrow K)\mp\sigma_{P(p \rightarrow K)} \\
  P(\pi \rightarrow p)\mp\sigma_{P(\pi \rightarrow p)} & P(K \rightarrow p)\mp\sigma_{P(K \rightarrow p)} & P(p \rightarrow p)\pm\sigma_{P(p \rightarrow p)}
  \end{bmatrix}
\end{equation}

The raw multiplicities $M^{h^{\pm},+}_{raw}$ and $M^{h^{\pm},-}_{raw}$ are then recalculated using the altered probability matrices. The largest difference between
$M^{h^{\pm},+}_{raw}$ and $M^{h^{\pm},-}_{raw}$ with $M^{h^{\pm}}_{raw}$ is taken as the sytematic error :

\begin{equation}
  \sigma^{RICH_stat}_{sys} = MAX(|M^{h^{\pm},+}_{raw}-M^{h^{\pm}}_{raw}|,|M^{h^{\pm},-}_{raw}-M^{h^{\pm}}_{raw}|)
\end{equation}

The final systematic uncertainty associated to the particle identification and unfolding correction ($\sigma^{RICH}_{sys}$) is the largest value of $\sigma^{RICH_stat}_{sys}$
 and $\sigma^{RICH_LH}_{sys}$. The relative error $\sigma^{RICH}_{sys}$ is shown in Fig.\ref{}.
 + discuss results.

%----------------------------------------------------------------------------------------

\section{Data sets}

Content

%----------------------------------------------------------------------------------------

\section{Raw Multiplicities}

Content
