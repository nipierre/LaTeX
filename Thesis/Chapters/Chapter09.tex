% Chapter 9

\chapter{Analysis of 2016 raw multiplicities} % Chapter title

\label{ch:raw} % For referencing the chapter elsewhere, use \autoref{ch:name}

The 2006 COMPASS multiplicities results were based on data with a deuteron target ($^6$LiD)
and the results with these data are not constraining the strange quark fragmentation function.
With the analysis of new data with pure proton target (LH$_2$), the data obtained will introduce
a new set of equations linking the multiplicities with the fragmentation functions but still
invloving the same quark fragmentation functions we are interested in. The fact that the proton
and deuteron data can be fitted together will add constrains to the fragmentation function extraction.
In order to perform this king of study, one need a precision of 5 to 7\% on the multiplicities.

The analysis is performed on COMPASS data recorded in 2016 using a 160 GeV muon beam incident on a proton
target (LH$_2$). Five weeks of the 2016 data are analyzed.

%----------------------------------------------------------------------------------------

\section{Method of extraction}

The method of extraction of the multiplicities follow several steps. For each selection steps,
a list of cuts is applied on both geometrical and physical quantities. First the DIS events are selected
and then the SIDIS events (hadrons) are selected. For the DIS event selection, a study on the target radius
had to be made in order to determine the optimal value for the target cut. When the event selection is done,
the hadrons have to be identified between pions, kaons and protons and the identification has to be corrected
according to the RICH detection efficiency in a process called unfolding. The obtained raw multiplicities are
then binned and corrected with the detector acceptance. The general event reconstruction codes for COMPASS are
used. A personal analysis code is developed to study the SIDIS channel and select pion or kaon production events.

In the following, the number of residual events after each cut will be given between parentheses next to each cut.
When the number is not specified, it can mean that the cut was done in a pre-analysis of that the recovery of the
number is too complex due to the transversal implementation of the cut in the code (e.g. for the $\nu$ cut).

%------------------------------------------------

\section{DIS event selection}

For the DIS event selection ($\mu p \rightarrow \mu X$), the following list of cuts is applied :
\begin{enumerate}
  \item Events with Best Primary Vertex ( events)
  \item Events with reconstruted scattered muon ( events)
  \item Events with a well reconstructed beam track (so-called 'BMS cut') ( events)
  \item Events with energy of beam muon energy in range [140 GeV, 180 GeV] ( events)
  \item Events with primary interaction in the target material, target radius cut (explained in section ??, events)
  \item Events with muon beam trajectory crossing entirely the target cell ( events)
  \item Events with Middle, Ladder, Outer or LAST trigger ( events)
  \item Events with $Q^2>1$ (GeV/c)$^2$ (DIS validity, events)
  \item Events with $0.1 < y < 0.9$ ( events)
  \item Events with $5 < W < 17$ GeV/c$^2$ ( events)
  \item Events with $0.004 < x < 0.4$ ( events)
  \item Events abiding $\nu$ cut ( events)
\end{enumerate}

The cut on th kinematic variable $\nu$ was implemented to reject events that contain hadrons outside of the measured
momentum range of 3 - 40 GeV/c. The criteria is defined by :
\begin{equation}
  \nu_{max} = \frac{\sqrt{(p^2_{max}+m^2_h)}}{z_{max}}
\end{equation}
\begin{equation}
  \nu_{min} = \frac{\sqrt{(p^2_{min}+m^2_h)}}{z_{min}}
\end{equation}

where $p_{max}$ (resp. $p_{min}$) is the hadron momentum limit of 40 GeV/c (resp. 3 GeV/c), $z_{max}$ (resp. $z_{min}$)
is the upper (resp. lower) value of the $z$-bin and $m_h$ is the mass of the considered hadron.

%------------------------------------------------

\section{Target radius cut evaluation}

TBA

%------------------------------------------------

\section{Hadron selection}

For the hadron selection ($\mu p \rightarrow \mu hX$), the following list of cuts is applied :
\begin{enumerate}
  \item Particle is not a scattered muon ( hadrons)
  \item Z coordinate of the first measured hit < 350 cm ( hadrons)
  \item Z coordinate of the last measured hit > 350 cm ( hadrons)
  \item Maximum radiation length cumulated along all the trajectory < 15 radiation lengths ( hadrons)
  \item $3 < p_h < 40$ GeV/c ( hadrons)
  \item $0.01 < \theta_{RICH} < 0.12$ (at RICH entrance, hadrons)
  \item $x^2_{RICH} + y^2_{RICH} > 25$ cm$^2$ ( hadrons)
  \item $0.2 < z < 0.85$ ( hadrons)
\end{enumerate}

%----------------------------------------------------------------------------------------

\section{Particle Identification with RICH detector}

The $\pi$ and $K$ particle identification (PID) is performed by the RICH detector.

The method used for the RICH particle identification is described in \ref{}. The idea is the following : when a particle
is detected, six likelihood functions are calculated ($\pi$, $K$, $p$, $e$, $\mu$ and the background) and are then
compared to make the particle identification. The evaluation is done separately for pions, kaons and protons. The largest
value corresponds to the maximal probability. The method is improved by looking further to $LH(2^{nd})$ which is the second
highest value of the four compared likelihood values ($\pi$, $K$, $p$ and the background). The electron and muon likelihood
are not considered in the assignment of $LH(2^{nd})$ as in the chosen momentum range (2 to 40 GeV/c) the RICH detector can
not be used to efficiently distinguish electrons from $\pi$.

All $\pi$, $K$ and $p$ probabilities are needed for the unfolding.

\begin{enumerate}
  \item Pion selection
  \begin{itemize}
    \item $LH(\pi) > 0$
    \item $LH(\pi) > LH(K)$, $LH(p)$ and $LH(bgd)$. In case $LH(e) > 1.8LH(\pi)$, one must consider the electron hypothesis
    in the previous comparison.
    \item $\frac{LH(\pi)}{LH(2^{nd})}>1.02$
    \item $\frac{LH(\pi)}{LH(bgd)}>2.02$
  \end{itemize}
  \item Kaon selection
  \begin{itemize}
    \item $LH(K) > 0$
    \item $LH(K) > LH(\pi)$, $LH(p)$ and $LH(bgd)$. In case $LH(e) > 1.8LH(\pi)$, one must consider the electron hypothesis
    in the previous comparison.
    \item $\frac{LH(K)}{LH(2^{nd})}>1.08$
    \item $\frac{LH(K)}{LH(bgd)}>2.08$
  \end{itemize}
  \item Proton selection
  Three cases are considered depending on the momentum $p$ of the particle and are defined but the kaon threshold ($\simeq 8.9$ GeV/c)
  and proton threshold ($\simeq 17.95$ GeV/c)
  \begin{enumerate}[(a)]
    \item Kaon threshold $< p \leq$ proton threshold - 5 GeV/c
    \item $p >$ proton threshold + 5 GeV/c
    \begin{itemize}
      \item $LH(p) > 0$
      \item $LH(p) > LH(\pi)$, $LH(K)$ and $LH(bgd)$. In case $LH(e) > 1.8LH(\pi)$, one must consider the electron hypothesis
      in the previous comparison.
      \item $\frac{LH(p)}{LH(2^{nd})}>1$
    \end{itemize}
    \item Proton threshold - 5 GeV/c $< p <$ proton threshold + 5 GeV/c
    \begin{itemize}
      \item Using (a) and (b) simultaneously.
    \end{itemize}
  \end{enumerate}
\end{enumerate}

%----------------------------------------------------------------------------------------

\section{RICH unfolding based on efficiency matrices}

The performance of the RICH is not perfect : in terms of efficiency and purity, some particles
are misidentified.

The unfolding procedure is needed to correct the yield of identified hadrons for RICH detection efficiency.
In order to perform this correction, the RICH actual performance is evaluated from real data. The result of
this evaluation is presented through RICH efficiency probabilities matrices, $M_{RICH}$, binned in momentum
and angle :

\begin{itemize}
  \item $p$ {3,12,13,15,17,19,22,25,27,30,35,40} GeV/c
  \item $\theta$ {0.01,0.04,0.12} rad
\end{itemize}

The 3-by-3 matrices $M_{RICH}$ give a relation between the vector of true hadron $T_h$ and the vector of
identified hadron $I_h$

\begin{equation}
\begin{bmatrix}
I_{\pi} \\
I_K \\
I_p
\end{bmatrix}
=
\begin{bmatrix}
\epsilon(\pi \rightarrow \pi) & \epsilon(K \rightarrow \pi) & \epsilon(p \rightarrow \pi)\\
\epsilon(\pi \rightarrow K) & \epsilon(K \rightarrow K) & \epsilon(p \rightarrow K) \\
\epsilon(\pi \rightarrow p) & \epsilon(K \rightarrow p) & \epsilon(p \rightarrow p)
\end{bmatrix}
\begin{bmatrix}
T_{\pi} \\
T_K \\
T_p
\end{bmatrix}
\end{equation}

The coefficients of the $M_{RICH}$, $\epsilon{t \rightarrow i}$, are the probabilities that a true hadron
$t$ is identified as a hadron of type $i$. These probabilities have been determined as described in \ref{}.

By performing a matrix inversion, one can obtain the unfolded number of hadrons with the Eq.\ref{} :

\begin{equation}
  \overrightarrow{T_h} = M^{-1}_{RICH}\overrightarrow{I_h}
\end{equation}

where $M^{-1}_{RICH}$ coefficients are weights with which each identified hadron is counted as a pion, kaon
or proton.

\begin{table}
  \caption{}
  \label{}
  %Table of hadron counting
\end{table}

%----------------------------------------------------------------------------------------

\section{Errors on the RICH unfolding}

The RICH matrices are built using the statistical errors associated to the original probability matrix.

\begin{equation}
M^{\pm}_{RICH}
=
\begin{bmatrix}
\epsilon(\pi \rightarrow \pi)\pm\sigma_{\epsilon(\pi \rightarrow \pi)} & \epsilon(K \rightarrow \pi)\pm\sigma_{\epsilon(K \rightarrow \pi)} & \epsilon(p \rightarrow \pi)\pm\sigma_{\epsilon(p \rightarrow \pi)}\\
\epsilon(\pi \rightarrow K)\pm\sigma_{\epsilon(\pi \rightarrow K)} & \epsilon(K \rightarrow K)\pm\sigma_{\epsilon(K \rightarrow K)} & \epsilon(p \rightarrow K)\pm\sigma_{\epsilon(p \rightarrow K)} \\
\epsilon(\pi \rightarrow p)\pm\sigma_{\epsilon(\pi \rightarrow p)} & \epsilon(K \rightarrow p)\pm\sigma_{\epsilon(K \rightarrow p)} & \epsilon(p \rightarrow p)\pm\sigma_{\epsilon(p \rightarrow p)}
\end{bmatrix}
\end{equation}

As an inversion of the probabilities matrices is done in the analysis, the propagation of errors through the
inversion operation has to be performed. A calculation of the uncertainties on the probabilities yields \cite{} :

\begin{equation}
  [\sigma^{-1}_i]^2 = \epsilon^{-1}_{ip}\epsilon^{-1}_{ir}cov(\epsilon_{pq},\epsilon_{rs})\epsilon^{-1}_{qi}\epsilon^{-1}_{si} + [\epsilon^{-1}_{ik}\sigma_k]^2
\end{equation}

This equation leads to the inverse error matrices with associated statistical errors :

\begin{equation}
[M^{\pm}_{RICH}]^{-1}
=
\begin{bmatrix}
\epsilon^{-1}(\pi \rightarrow \pi)\pm\sigma^{-1}_{\epsilon(\pi \rightarrow \pi)} & \epsilon^{-1}(K \rightarrow \pi)\pm\sigma^{-1}_{\epsilon(K \rightarrow \pi)} & \epsilon^{-1}(p \rightarrow \pi)\pm\sigma^{-1}_{\epsilon(p \rightarrow \pi)}\\
\epsilon^{-1}(\pi \rightarrow K)\pm\sigma^{-1}_{\epsilon(\pi \rightarrow K)} & \epsilon^{-1}(K \rightarrow K)\pm\sigma^{-1}_{\epsilon(K \rightarrow K)} & \epsilon^{-1}(p \rightarrow K)\pm\sigma^{-1}_{\epsilon(p \rightarrow K)} \\
\epsilon^{-1}(\pi \rightarrow p)\pm\sigma^{-1}_{\epsilon(\pi \rightarrow p)} & \epsilon^{-1}(K \rightarrow p)\pm\sigma^{-1}_{\epsilon(K \rightarrow p)} & \epsilon^{-1}(p \rightarrow p)\pm\sigma^{-1}_{\epsilon(p \rightarrow p)}
\end{bmatrix}
\end{equation}

%----------------------------------------------------------------------------------------

\section{Kinematic binning}

The raw multiplicities are evaluated in bins of the Bjorken variable $x$, the muon energy fraction carried
by the virtual photon $y$ and the virtual photon energy fraction carried by final state hadron $z$. They
are calculated with the following formula :

\begin{equation}
  \frac{dM^h(x,y,z)}{dz}=\frac{1}{N^{DIS}_{Events}(x,y)}\frac{dN^{DIS}_{h}(x,y,z)}{dz}
\end{equation}

where $N^{DIS}_{Events}$ is the number of DIS events and $N^{DIS}_{h}$ is the number of
hadrons after RICH unfolding. As in practise, the multiplicities are measured in bins of
x (9 bins), y (6 bins) and z (12 bins), the calculated multiplicities can be expressed as :

\begin{equation}
  M^h_{raw}(x,y,z) = \frac{N^{DIS}_{h}(x,y,z)/\delta z}{N^{DIS}_{Events}}
\end{equation}

where $\delta z$ is the width of the z bin. For the multiplicities extraction, the binning in
$x$, $y$ and $z$ is the following :

\begin{itemize}
  \item $x$ {0.004,0.01,0.02,0.03,0.04,0.06,0.1,0.14,0.18,0.4}
  \item $y$ {0.1,0.15,0.2,0.3,0.5,0.7,0.9}
  \item $z$ {0.2,0.25,0.3,0.35,0.4,0.45,0.5,0.55,0.6,0.65,0.7,0.75,0.85}
\end{itemize}

%----------------------------------------------------------------------------------------

\section{Detector acceptance}

The COMPASS detector does not cover the full phase-space then the measured multiplicities have
to be corrected for the finite detector acceptance of the order of 70\%. The correction is
done using a Monte-Carlo dataset containing about 400 million events generated in the kinematic
region $Q^2 > 0.8$ (GeV/c)$^2$, x $\in$ [10$^{-4}$], y $\in$ [0.05,0.95]. It is determined as the ratio
of reconstructed multiplicities $M^h_r$ over the generated multiplicities $M^h_g$ and is binned in $x$, $y$
and $z$ :

\begin{equation}
  A^h(x,y,z) = \frac{M^h_r(x,y,z)}{M^h_g(x,y,z)}=\frac{N^h_r(x,y,z)/N^{DIS}_r(x,y,z)}{N^h_g(x,y,z)/N^{DIS}_g(x,y,z)}
\end{equation}

The correction is then applied to the raw multiplcities :

\begin{equation}
  M^h(x,y,z) = \frac{M^h_{raw}(x,y,z)}{A^h(x,y,z)}
\end{equation}

%----------------------------------------------------------------------------------------

\section{Data sets}

Content

%----------------------------------------------------------------------------------------

\section{Raw Multiplicities}

Content
