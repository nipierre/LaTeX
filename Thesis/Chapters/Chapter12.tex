% Chapter 11

\chapter{Final charged hadrons multiplicities} % Chapter title

\label{ch:mult} % For referencing the chapter elsewhere, use \autoref{ch:name}

%----------------------------------------------------------------------------------------


\section{Errors on the RICH unfolding}

The RICH matrices are built using the statistical errors associated to the original probability matrix.

\begin{equation}
M^{\pm}_{RICH}
=
\begin{bmatrix}
\epsilon(\pi \rightarrow \pi)\pm\sigma_{\epsilon(\pi \rightarrow \pi)} & \epsilon(K \rightarrow \pi)\pm\sigma_{\epsilon(K \rightarrow \pi)} & \epsilon(p \rightarrow \pi)\pm\sigma_{\epsilon(p \rightarrow \pi)}\\
\epsilon(\pi \rightarrow K)\pm\sigma_{\epsilon(\pi \rightarrow K)} & \epsilon(K \rightarrow K)\pm\sigma_{\epsilon(K \rightarrow K)} & \epsilon(p \rightarrow K)\pm\sigma_{\epsilon(p \rightarrow K)} \\
\epsilon(\pi \rightarrow p)\pm\sigma_{\epsilon(\pi \rightarrow p)} & \epsilon(K \rightarrow p)\pm\sigma_{\epsilon(K \rightarrow p)} & \epsilon(p \rightarrow p)\pm\sigma_{\epsilon(p \rightarrow p)}
\end{bmatrix}
\end{equation}

As an inversion of the probabilities matrices is done in the analysis, the propagation of errors through the
inversion operation has to be performed. A calculation of the uncertainties on the probabilities yields \cite{} :

\begin{equation}
  [\sigma^{-1}_i]^2 = \epsilon^{-1}_{ip}\epsilon^{-1}_{ir}cov(\epsilon_{pq},\epsilon_{rs})\epsilon^{-1}_{qi}\epsilon^{-1}_{si} + [\epsilon^{-1}_{ik}\sigma_k]^2
\end{equation}

This equation leads to the inverse error matrices with associated statistical errors :

\begin{equation}
[M^{\pm}_{RICH}]^{-1}
=
\begin{bmatrix}
\epsilon^{-1}(\pi \rightarrow \pi)\pm\sigma^{-1}_{\epsilon(\pi \rightarrow \pi)} & \epsilon^{-1}(K \rightarrow \pi)\pm\sigma^{-1}_{\epsilon(K \rightarrow \pi)} & \epsilon^{-1}(p \rightarrow \pi)\pm\sigma^{-1}_{\epsilon(p \rightarrow \pi)}\\
\epsilon^{-1}(\pi \rightarrow K)\pm\sigma^{-1}_{\epsilon(\pi \rightarrow K)} & \epsilon^{-1}(K \rightarrow K)\pm\sigma^{-1}_{\epsilon(K \rightarrow K)} & \epsilon^{-1}(p \rightarrow K)\pm\sigma^{-1}_{\epsilon(p \rightarrow K)} \\
\epsilon^{-1}(\pi \rightarrow p)\pm\sigma^{-1}_{\epsilon(\pi \rightarrow p)} & \epsilon^{-1}(K \rightarrow p)\pm\sigma^{-1}_{\epsilon(K \rightarrow p)} & \epsilon^{-1}(p \rightarrow p)\pm\sigma^{-1}_{\epsilon(p \rightarrow p)}
\end{bmatrix}
\end{equation}

%----------------------------------------------------------------------------------------

\section{Summary of systematic studies}

Content

%------------------------------------------------

\subsection{Subsection Title}

Content

%------------------------------------------------

\subsection{Subsection Title}

Content

%----------------------------------------------------------------------------------------

\section{Multiplicities ($h^{\pm}$,$\pi^{\pm}$,$K^{\pm}$)}

Content
