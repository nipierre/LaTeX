% Chapter X

\chapter{Summary and Conclusion} % Chapter title

\label{ch:CCL} % For referencing the chapter elsewhere, use \autoref{ch:name}

%----------------------------------------------------------------------------------------

In order to improve the knowledge on quark fragmentation functions into charged hadrons, the production of charged hadrons in semi-inclusive deep inelastic scattering on a pure proton target (lH$_2$) was studied.

The multiplicities of charged inidentified hadrons, pions, kaons and protons have been determined in a binning of three kinematical variables : $x$, the fraction of momentum of the nucleon held by the struck quark, $y$, the fraction of energy of the incoming lepton held by the virtual photon and $z$, the fraction of energy of the virtual photon transferred to the observed hadron $h$. The measurement was performed using 5 periods (weeks) of 2016 COMPASS data ($\sim$ 8.3 $\times$ 10$^6$ DIS events and $\sim$ 2.6 $\times$ 10$^6$ hadrons). The kinematic domain covered by these data is the following : $Q^2$ > 1 (GeV/$c$)$^2$, $y$ $\in$ [0.1,0.7], $x$ $\in$ [0.004,0.4], $W$ $\in$ [5,17] GeV, and $z$ $\in$ [0.2,0.85].

The hadron identification as pion, kaon and proton was provided by the Ring Imaging CHerenkov (RICH) detector. The performance of this detector was determined using the same data used in the aforementioned analysis. Having a good knowledge of the RICH response is primordial to calculate the absolute value of the pion, kaon and proton multiplicities. The hadron momentum range is restricted to [12,40] GeV/$c$ and the polar angle in the range [0.01,0.12] rad in order to operate in a region where the RICH can fully discriminate pion, kaon and proton. The lower cut in hadron momentum is used to be above the kaon identification threshold ($\sim$ 9.45 GeV) while the upper limit is to avoid region where effects arising from saturation ($\beta \rightarrow 1$) begin to appear, in order to ensure a good charged hadron separation. The polar angle range is chosen so that the efficiencies are generally high and precisely measured. In this phase-space, high identification efficiencies are found for pions ($>$ 97\%), kaons ($>$ 90\%) and protons ($>$ 90\%) with low misidentifications probabilities ($<$ 5\% for pions, $<$ 7\% for kaons, $<$ 10\% for protons).

The hadrons multiplicities were also corrected for the geometrical limitation of the spectrometer and the data reconstruction efficiency. The global acceptance correction is estimated using a MC simulation. A new MC generator, DJANGOH\cite{DJANGOH,DJANGOHnote}, was studied both for generating radiative events in the MC simulation and computing radiative correction to the multiplicities in a three dimensional binning in $x$, $y$ and $z$. For the first time we have a solid determination of radiative effects dependent on the $z$ variable. DJANGOH gives compatible results when compared to radiative corrections obtained with other software (viz. TERAD) and offers solid performances in the comparison between data and MC. The contribution to the hadron yield from the vector meson production was also estimated.

The sum and ratio of charged hadrons are of special interest as they are integrated quantities and can be compared to results from other experiments. Concerning the ratios of multiplicities, the results from COMPASS for a proton target are as expected with respect to the results from COMPASS for an isoscalar target\cite{COMPASS2006Pi,COMPASS2006K}. A discrepancy is found with results from HERMES\cite{HERMESMult} for kaons, confirming the discrepancy already found for deuteron/isoscalar targets, but might be explained by the different kinematics of the data points of the two experiments and possible hadron mass corrections\cite{Accardi}. Concerning the sums of multiplicities, in the $x$ region not concerned by a possible acceptance problem, the results from COMPASS for a proton target are compatible with the expections drawn from results for an isoscalar target. Again, the COMPASS results differ with the HERMES ones on this quantity for both pions and kaons, which might be explained by the fact that HERMES operates at lower $\nu$ than COMPASS, hence there might be phase space limitation effects in play\cite{MarcinPubli}.

The $K^+$ and $K^-$ multiplicities were used to extract the favoured $D^{K}_{fav}$, unfavoured $D^{K}_{unf}$ and strange $D^{K}_{s}$ quark fragmentation with a fit at LO, assuming the PDFs known.

The pion, kaon and proton multiplicity sets obtained in this analysis, which represent in total more than 1800 data points, are a major input for the global fit of world data done at NLO. The kaon multiplicity set is particularly awaited as it will enlarge significantly the available data set\cite{DIS2019}. The proton multiplicity set is also a novelty which will interest the fitters.
