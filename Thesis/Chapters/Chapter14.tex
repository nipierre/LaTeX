% Chapter 15

\chapter{Summary and Conclusion} % Chapter title

\label{ch:CCL} % For referencing the chapter elsewhere, use \autoref{ch:name}

%----------------------------------------------------------------------------------------

In order to improve the knowledge on quark fragmentation into charged hadrons, the production of charged hadrons in deep inelastic scattering on a pure proton target (lH$_2$) was studied in a semi-inclusive measurement.

The multiplicities of charged unidentified hadrons, pions, kaons and protons have been determined in a binning of three kinematic variables: $x$, the fraction of momentum of the nucleon held by the struck quark, $y$, the fraction of energy of the incoming lepton held by the virtual photon and $z$, the fraction of energy of the virtual photon transferred to the observed hadron $h$. The measurement was performed using 5 periods (weeks) of $2016$ COMPASS data ($\sim$ $8.3$ $\times$ $10^6$ DIS events and $\sim$ $2.6$ $\times$ $10^6$ hadrons). The kinematic domain covered by these data is the following: $Q^2$ $>$ $1$ (GeV/$c$)$^2$, $y$ $\in$ [$0.1,0.7$], $x$ $\in$ [$0.004,0.4$], $W$ $\in$ [$5,17$] GeV/$c^2$, and $z$ $\in$ [$0.2,0.85$].

The hadron identification as pion, kaon and proton was provided by the Ring Imaging CHerenkov (RICH) detector. The performance of this detector was determined using the same data used in the aforementioned analysis. Having a good knowledge of the RICH response is mandatory to calculate pion, kaon and proton multiplicities. Due to the usage of the RICH, the hadron momentum range is restricted from $12$ GeV/$c$ to $40$ GeV/$c$ and the polar angle in the range from $0.01$ to $0.12$ rad in order to operate in a region where the RICH can fully discriminate pions, kaons and protons. With the lower cut in hadron momentum it is ensured that the particle momenta are well above the kaon identification threshold (about $9.45$ GeV), while with the upper limit regions are avoided, where effects arising from saturation ($\beta \rightarrow 1$) begin to appear, in order to ensure a good charged hadron separation. The polar angle range is chosen, so that the RICH efficiencies are generally high and precisely measured. In this phase-space, high identification efficiencies are found for pions ($>$ $97$\%), kaons ($>$ $95$\%) and protons ($>$ $90$\%) with low misidentifications probabilities ($<$ $5$\% for pions, $<$ $7$\% for kaons, $<$ $10$\% for protons).

The hadrons multiplicities were also corrected for the geometric limitations of the spectrometer, detector performance and the data reconstruction efficiency. The global acceptance correction is estimated using a MC simulation. A new MC generator, DJANGOH \cite{DJANGOH,DJANGOHnote}, was applied both for generating radiative events in the MC simulation and computing radiative correction to the multiplicities in a three dimensional binning in $x$, $y$ and $z$. For the first time we have a solid determination of radiative effects dependent on the $z$ variable. DJANGOH gives compatible results when compared to radiative corrections obtained with TERAD (analytic calculation) and gives solid results about the electron production from photons. The contribution to the hadron yield from the vector meson production was also estimated.

The sum and ratio of charged hadrons are of special interest as they are integrated quantities and can be compared to results from other experiments. For the ratios of multiplicities, the results from COMPASS for a proton target are as expected with respect to the results from COMPASS for an isoscalar target \cite{COMPASS2006Pi,COMPASS2006K}. A discrepancy is found with results from HERMES \cite{HERMESMult} for kaons, confirming the discrepancy already found for deuteron/isoscalar targets, but are explained by the different kinematics of the data points of the two experiments \cite{MarcinPubli}. For the sums of multiplicities, in the $x$ region not affected by a possible acceptance problem, the results from COMPASS for a proton target are compatible with the LO pQCD expections. Again, the COMPASS results differ with the HERMES ones on this quantity for both pions and kaons, which might be explained by the fact that HERMES operates at lower $\nu$ than COMPASS and possible hadron mass corrections \cite{Accardi,MarcinPubli}. The concept of FFs is then not applicable in part of the phase space \cite{MarcinPubli}.

The $K^+$ and $K^-$ multiplicities were used to extract the favoured $D^{K}_{fav}$, unfavoured $D^{K}_{unf}$ and strange $D^{K}_{s}$ quark fragmentation with a fit at LO, assuming the PDFs known. The result of the fit points out that there is a bad sensitivity to the strange quark of these measurements. The fit gives too much contribution to the favoured and unfavoured fragmentation functions at the expense of the strange fragmentation function.

The pion, kaon and proton multiplicity sets obtained in this analysis, which represent in total more than 1800 data points, are a major input for the global fit of world data done at NLO. The kaon multiplicity set is particularly awaited as it will enlarge significantly the available data set \cite{DIS2019}. The proton multiplicity set is also a novelty which will interest the fitters.

This analysis will be pursued inside COMPASS. The new RICH calibration will improve the identification efficiency of charged hadrons at high momenta, while the inclusion of more data from $2016$ and data from $2017$ will improve the statistics. Further study on the Monte Carlo and the comparison between data and Monte-Carlo will improve the systematics. As soon as the problem of the drop in $x$ of the charged pion and charged hadron sums is solved, the extraction of the pion fragmentation functions can be done. In contrast to the kaons, the extraction of the pion fragmentation functions with a LO fit of the pion multiplicities is more likely to converge as the measurement have a good sensitivity of the $u$ and $d$ quark, hence the favoured and unfavoured pion fragmentation functions. For kaons, the extraction of $K^0$ multiplicities from data could help constrain the kaon fragmentation function fit. This analysis has already started. In addition to the data taken in $2016$ and $2017$, more data will be taken in $2021$ on deuteron target. Eventually as it was shown that our kaon and proton multiplicity ratios at $z$ above $0.7$ are not described by pQCD \cite{MarcinPubli}, it would be interesting to take it into account in our analysis and further investigate this kinematic region.
