% Chapter 10

\chapter{Correction factors to the multiplicities} % Chapter title

\label{ch:CF} % For referencing the chapter elsewhere, use \autoref{ch:name}

%----------------------------------------------------------------------------------------

\section{Detector acceptance}

The COMPASS detector does not cover the full phase-space then the measured multiplicities have
to be corrected for the finite detector acceptance of the order of 70\%. The correction is
done using a Monte-Carlo dataset containing about 400 million events generated in the kinematic
region $Q^2 > 0.8$ (GeV/c)$^2$, x $\in$ [10$^{-4}$], y $\in$ [0.05,0.95].

The events are created with DJANGOH generator with parametrization of the parton distribution functions
(MSTW08). In addition, the use of JETSET inside DJANGOH allows the hadronization of quarks q to final-state
hadrons h according to the Lund model. The COMPASS high $p_T$ tuning was used, resulting in a good description
of real data as shown for DIS and hadrons in Fig. \ref{}.

The same DIS event and unidentified hadron selection that are used on real data (except the BMS cut) are applied
to the MC data sample for reconstructed MC events and particles.

The data are processed through a GEANT4 model of the spectrometer, TGEANT, and events are reconstructed with the
same CORAL version as for the real data.

The acceptance involves both reconstructed and generated particles. In both cases, the particle ID is taken from
the MC truth. The following selection is made on the generated events and particles :

\begin{enumerate}
  \item Energy of the beam muon in range [140,180] GeV
  \item Z coordinate of event vertex ($z_{vtx}$) within the target region
  \item $Q^2>1$ (GeV/c)$^2$
  \item $0.1 < y < 0.9$
  \item $0.004 < x < 0.4$
  \item $5 < W < 17$ GeV/c$^2$
  \item $\nu$ range used in data
  \item $0.2 < z < 0.85$
\end{enumerate}

In the following, $r$ and $g$ refers to 'reconstructed' and 'generated' quantities.

The acceptance is determined as the ratio of reconstructed multiplicities $M^h_r$ over the generated multiplicities $M^h_g$
and is binned in $x$, $y$ and $z$ :

\begin{equation}
  A^h(x,y,z) = \frac{M^h_r(x,y,z)}{M^h_g(x,y,z)}=\frac{N^h_r(x,y,z)/N^{DIS}_r(x,y,z)}{N^h_g(x,y,z)/N^{DIS}_g(x,y,z)}
\end{equation}

where $x_g$, $y_g$ and $z_g$ are the generated kinematic values and $x_r$, $y_r$ and $z_r$ are the reconstructed kinematic
values. Used in this fashion, the kinematic bin smearing due to reconstruction limitations is accounted for. A more rigorous
bin smearing correction would involve an unfolding procedure but is not done in this analysis.

For this method, the error estimation is difficult to rigorously calculate as the numbers of evaluated hadrons and DIS events,
in both the reconstructed and generated case, are not independent. An estimation is made by considering that the hadrons numbers
and DIS events are independent of each other.

Due to the $z$ kinematic bin migration effects, there exist particles in $N_r$ which are independent from $N_g$. Decomposing $N_r$
into two independent samples namely $N_{r^0}$ which are contained in $N_g$ and $N_{r'}$ which are not, the final acceptance error yields :

\begin{equation}
  \begin{split}
    E^2_{acc} = \left (\frac{G_D}{R_D+R'_{D}}\right )^2\left [\frac{(R_h+A)(G_h-R_h+1)}{(G_h+2)^2(G_D+3)}+\frac{R'_{h}}{G^2_h}+\frac{R'^2_h}{G^3_h}\right ] \\
                + \left (\frac{G_D}{R_D+R'_{D}}\right )^4\left (\frac{R_h+R'_h}{G_h}\right )^2\left [\frac{(R_D+1)(G_D-R_D+1)}{(G_D+2)^2(G_D+3)}+\frac{R'_D}{G^2_D}+\frac{R'^2_D}{G^3_D}\right ]
  \end{split}
\end{equation}

where $G_h$ (resp. $G_D$) are the generated hadrons (resp. DIS events) in a given $x$, $y$, $z$ bin, $R_h$ (resp. $R_D$) the reconstructed
hadrons (resp. DIS events) and $R'_h$ (resp. $R'_D$) all other particles (resp. events) that are reconstructed as hadrons (resp. DIS events)
in a given $x$, $y$, $z$ bin.

The correction is then applied to the raw multiplcities :

\begin{equation}
  M^h(x,y,z) = \frac{M^h_{raw}(x,y,z)}{A^h(x,y,z)}
\end{equation}

%----------------------------------------------------------------------------------------

\section{Diffractive vector meson correction}

It is usually assumed that hadrons produced in SIDIS originate from lepton-parton scattering. Nevertheless the scattering of a lepton
off a nucleon can also result in the diffractive production of vector mesons. These particles decay into lighter mesons that cannot be
distinguished from the one resulting from the hadronization of a quark originating from the target nucleon. This implies that fragmentation
functions extracted from multiplicities contaminated with diffractive vector mesons would violate universality, as they would be process
dependent. However, this is a complex theoretical discussion so the multiplicities both with and without subtracting the diffractive vector
meson contribution are calculated as well as the separate correction factors for DIS events and hadrons.

For kaons, the dominant vector meson contribution comes from the diffractive production of $\rho^0$ and $\Phi$ :
\begin{equation}
    \gamma * p \rightarrow \rho^0 p \rightarrow p\pi^+\pi^-
    \gamma * p \rightarrow \Phi p \rightarrow pK^+K^-
\end{equation}

This process is mainly exclusive but in 20\% of cases a diffractive dissociation of the target nucleon occurs. Other channels (excited $\rho$, $\omega$, etc.)
are expected to contribute much less and are not taken into account. As pions and kaons stemming from diffractive
vector meson decay cannot be separated from the one resulting from SIDIS, the evaluation of their contribution to the multiplicities is based on a
Monte Carlo study. Three Monte Carlo samples are produced based on different generators (SIDIS using DJANGOH, diffractive $\Phi$ using HEPGEN++) and
the same event reconstruction chain. For the diffractive vector meson samples, both exclusive events and events with diffractive dissociation of the
proton are simulated. The $\rho^0$ sample includes nuclear effects (coherent production and nuclear absorption).

The fraction of pions (resp. kaons) resulting from a diffractive $rho^0$ (resp. $\Phi$) is calculated in the same binning as the raw multiplicities as :

\begin{equation}
  \begin{split}
    f^{\pi}_{\rho^0}(x,y,z) = \frac{N^{\pi}_{HEPGEN++}(x,y,z)}{N^{\pi}_{DJANGOH}(x,y,z)+N^{\pi}_{HEPGEN++}(x,y,z)} \\
    f^K_{\Phi}(x,y,z) = \frac{N^K_{HEPGEN++}(x,y,z)}{N^K_{DJANGOH}(x,y,z)+N^K_{HEPGEN++}(x,y,z)}
  \end{split}
\end{equation}

where $N^{\pi}_{HEPGEN++}$, $N^{\pi}_{DJANGOH}$, $N^K_{HEPGEN++}$ and $N^K_{DJANGOH}$ are the number of kaons reconstructed from the HEPGEN++ and DJANGOH MC samples normalized by the corresponding
MC luminosity ($L_{MC}$). The luminosity depends on the event weighting and the process cross-section $\sigma_{int}$ (DIS for DJANGOH event and diffractive
vector meson production for HEPGEN++ events). The final weighted number of kaons is summarized in Table \ref{}.

\begin{equation}
  \sum_{events} w_i = L_{MC} \cdot \sigma{int}
\end{equation}

\begin{table}
  \caption{}
  \label{}

\end{table}

The diffractive vector meson events can also lead to a contamination in DIS events. Here, the two channels studied are diffractive $\rho^0$ and $\Phi$
with the fraction of the contamination expressed in Eqs. \ref{}. Contrary to previous Eq. \ref{}, the denominator only includes the DIS events from the
DJANGOH generator because the cross-section used to generate the DJANGOH sample takes into account the diffractive contribution.

\begin{equation}
  \begin{split}
    f^{\rho^0}_{DIS}(x,y,z) = \frac{N^{DIS}_{\rho^0,HEPGEN++}(x,y,z)}{N^{DIS}_{DJANGOH}(x,y,z)} \\
    f^{\Phi}_{DIS}(x,y,z) = \frac{N^{DIS}_{\Phi,HEPGEN++}(x,y,z)}{N^{DIS}_{DJANGOH}(x,y,z)}
  \end{split}
\end{equation}

The total contribution from the diffractive vector-meson contribution ($f^{VM}_{DIS}$) to the DIS sample is the sum of the $f^{\rho^0}_{DIS}$ and $f^{\Phi}_{DIS}$.
The final correction reads as follows :

\begin{equation}
  \begin{split}
  B^h(x,y,z) = \frac{ \frac{N^{\pi}(x,y,z)}{N^h(x,y,z)}\left (1-f^{\pi}_{\rho^0}(x,y,z)\right )
                   + \frac{N^K(x,y,z)}{N^h(x,y,z)}\left (1-f^{K}_{\Phi}(x,y,z)\right ) + \frac{N^p(x,y,z)}{N^h(x,y,z)} }{1-f^{VM}_{DIS}(x,y,z)} \\
  B^{\pi}(x,y,z) = \frac{1-f^{\pi}_{\rho^0}(x,y,z)}{1-f^{VM}_{DIS}(x,y,z)} \\
  B^K(x,y,z) = \frac{1-f^{K}_{\Phi}(x,y,z)}{1-f^{VM}_{DIS}(x,y,z)}
  \end{split}
\end{equation}

%----------------------------------------------------------------------------------------

\section{Electron contamination}

The pion (and thus hadron) sample is contaminated by electrons and positrons. With the DJANGOH event generator, we are able to describe almost correctly the electron production from radiative photons : in Fig. \ref{electroprod} one can see that the discrepancy of electroproduction along $\Phi$ in the hadron production plane differ by 10\% at low $\Phi$.

Both in data and Monte-Carlo the pions and electrons are not separated. In doing so, the correction for electron contamination is performed by the acceptance, assuming that we describe properly the electron production in the data. In a discussion afterwards on the multiplicity sum of pions and hadrons, it seems that we are correcting only partially for this contamination.
