% Chapter 6

\chapter{DJANGOH} % Chapter title

\label{ch:DJANGOH} % For referencing the chapter elsewhere, use \autoref{ch:name}

%----------------------------------------------------------------------------------------

\section{Features of DJANGOH}

\subsection{Presentation of DJANGOH}
A quick summary of the DJANGOH generator can be done as following :
\begin{itemize}
\item DJANGOH\cite{DJANGOH} is at first a Monte-Carlo event simulation tool for neutral and charged current
$ep$ interactions at HERA with the event generators HERACLES and DJANGO6.
\item DJANGOH was then modified to also simulate $\mu p$ interactions at the COMPASS experiment.
\item The emphasis is put on the inclusion of QED radiative corrections
(single photon emission from the lepton or the quark line, self energy correction, complete
set of one-loop weak corrections). The background from radiative elastic scattering
$\mu p\rightarrow \mu p\gamma$ is also included.
\item HERACLES is treating the $lp$ scattering by means of structure function parametrizations
or parton distribution functions in the quark-parton model framework.
\item DJANGO6 is simulating deep inelastic scattering including both QED and QCD radiative effects.
\item DJANGOH is an interface to LEPTO\cite{LEPTO}, ARIADNE\cite{ARIADNE} (for parton cascades), PYTHIA\cite{PYTHIA6} (LUND
string fragmentation in JETSET\cite{JETSET} for hadronic final state) and SOPHIA\cite{SOPHIA} (for low-mass
hadronic final states).
\end{itemize}
DJANGOH is able to perform :
\begin{itemize}
\item Generation of $lp$ scattering with and without fragmentation for the final state with radiative events.
\item Calculation of cross-sections (radiative, born)
\item Calculation of radiative correction factors (inclusive, semi-inclusive)
\item Generation of event as an event generator in a Monte-Carlo chain
\end{itemize}

\subsection{Technical description of DJANGOH}

The computational procedures applied in DJANGOH are based on the methods used in AXO\cite{AXO}
library for Monte-Carlo integration and event generation. AXO relies on the Monte-Carlo
integration algorithm VEGAS\cite{VEGAS}. The computation is made in this order :

\begin{itemize}
\item Integration of the different contributions : partial cross-sections are determined
according to the defined phase-space region. They give the relative weight of the
corresponding contribution in the final step of event sampling. Moreover, the integration
procedure supplies information for the construction of the distribution function
applied for an event generation.
\item Estimation of the local maxima of the distribution function in a predefined number
of hypercubes.
\item According to the partial cross-sections that were calculated, events are generated
randomly from the individual contributions. HERACLES is only taking care of the scattered
lepton and the potential radiative photon. DJANGO is simulating the QCD effect and generates
the hadronic part of the event.
\end{itemize}


%----------------------------------------------------------------------------------------

\section{Consistency checks}

In order to test the self-consistency of DJANGOH, I generated a certain number of
events of $\mu p$ scattering with an incoming muon energy of 160 GeV. In Fig. \ref{fig:edist}),
the energy of the radiated photon (if one is present), the outgoing muon and the struck quark
are shown. The cutoff at low energy is given by the specified kinematic cuts in DJANGOH.

\begin{figure}[htb]
\centerline{\epsfig{file=gfx/Edist.png,width=17cm}}
\caption{Energy distribution for, from left to right, radiative photon, outgoing lepton and struck quark.
The emission of low energy radiative photon is priviledged by DJANGOH, while the rest of the energy is distributed
between the outgoing muon and the struck quark.}\label{fig:edist}
\end{figure}

We expect naively a peak around $E_{\gamma} = 0$ as soft photons (low energy photons)
are more likely to be emitted than hard photons (high energy photons). The rest of
the energy of the incoming muon is distributing accordingly between outgoing muon and
struck quark.

Fig. \ref{fig:anglesp} shows the $\theta_{\gamma}$ and $\theta_{\mu}$
distributions. As discussed before, the radiated photon has two priviledged direction of
emission, namely the s-peak and the p-peak, colinear to the direction of propagation
of the incoming and outgoing muons. In Fig. \ref{fig:anglesp} , the two distribution are plotted next
to each other, enabling to see if the p-peak is matching with the position of the peak in the scattering
angle of the outgoing muon, which is the case. The s-peak is around 0, which is
also expected.

A last exercise that I have done is to see whether DJANGOH, when we put a $Q^2>1$ constraint on event
generation, is producing radiative events with $Q^2<1$, which would be the case if a radiative photon
was emitted by the incoming lepton. A quick proof of this can be made starting from the relation
between $Q^2_{lep}$ and $Q^2_{had}$ :

\[Q^2_{had}=Q^2_{lep}+2E_\gamma(\nu_{lep}-\sqrt{\nu_{lep}^2+Q^2_{lep}}cos\theta_\gamma)\]

When a radiative photon is emitted by the incoming muon, $cos\theta_\gamma \simeq 1$ then
$\nu_{lep}-\sqrt{\nu_{lep}^2+Q^2_{lep}}cos\theta_\gamma) \leq 0$ leading to $Q^2_{had} \leq Q^2_{lep}$.
With an analoguous reasoning, if a radiative photon is emitted by the outgoing photon, then $Q^2_{had} \geq Q^2_{lep}$.
In Fig. \ref{fig:Q2corr}, $Q^2_{had}$ is shown as a function of $Q^2_{lep}$, for $Q^2_{lep}=1$. Values of $Q^2_{had}$ can be found
both above and below $Q^2=1$, which was the point to be verified. Another information that is given by this
plot is that as the distribution around $Q^2_{had} = Q^2_{lep}$ is narrow, most radiative photons are soft, ie. low
energetic, as noted previously.

\begin{figure}[htb]
\centerline{\epsfig{file=gfx/anglesp.png,width=16cm}}
\caption{$\theta$ angle distribution for the radiative photon (up) and the outgoing muon (down). From left
to right the scaling of the x-axis is changed (normal, logarithmic, logarithmic with rebinning).
One can see two peaks in the theta distribution of the radiative photon : one around zero (s-peak) and
one a little bit further (p-peak). When compared to the $\theta$ distribution of the outgoing muon, especially with the
last scaling, one can see that the two peaks match.}\label{fig:anglesp}
\end{figure}
\hfill
\begin{figure}[htb]
\centerline{\epsfig{file=gfx/Q2_corr.png,width=16cm}}
\caption{$Q^2$ correlation plot $Q^2_{had}=f(Q^2_{lep})$. On the left is the plot for the complete range of $Q^2$
given by the kinematic constraints. The scattering around the $Q^2_{had} = Q^2_{lep}$ line is small, indicating that
most of the radiative photons are soft. On the right is the same plot but restricted to the $Q^2_{lep}\in[1,2]$ and
$Q^2_{had}\in[0,2]$ region. The fact is that for $Q^2_{lep}=1$, $Q^2_{had}$ takes values in above and below $Q^2=1$, as expected}\label{fig:Q2corr}
\end{figure}

%----------------------------------------------------------------------------------------

\section{Enhancement of DJANGOH}

An upgrade to the original DJANGOH is the possibility to use different input energies for
the incoming lepton for mutiple event generation. It was before only allowed to
specify one input energy at the launch of the program. DJANGOH is now capable to take
into account a new beam energy at each new event.
Nonetheless, using different input energies for event generation is causing a problem :
the cross-sections that are needed for event generation are depending on this input energy.
The naive way would be to recompute the cross-section for each event but this solution
takes much time due to the computation of the cross-section being the slowest part of
the generation.

\begin{figure}[htb]
\centerline{\epsfig{file=gfx/gridxs.png,width=16cm}}
\caption{Cross-section values (in nB) for different bins in a range for the energy of the incoming muon between 140 and 180 GeV for Virtual/Soft cross-section
(gray blue), Initial State Radiation cross-section (green), Final State Radiation cross-section (yellow), Compton contribution
cross-section (orange) and the total cross-section (red). The cross-section distribution is almost flat, with an overall difference
between the first and last bin in energy of 13.9 nB for a mean total cross-section of about 500 nB. Though it represents only a 2 percent
change in the cross-section, it has to be taken into account for a proper event generation.}\label{fig:gridxs}
\end{figure}

\newpage

A solution to circumvent this problem is the use
of a grid of cross-sections. This grid is initialized after a rough specification of the
type of dispersion in energy of the considered beam.
Basically, the grid needs to have
a mean energy and the standard deviation of energy to this mean energy, as well as the
number of bins in the grid. Then for each bin, the energy of the center of the bin is taken
and the cross-section corresponding to this energy is computed. The narrower the bins are
the more accurate the cross-section is for the considered bin. As shown in Fig. \ref{fig:gridxs}, with a
mean energy of 160 GeV, a distribution width of 20 GeV and 20 bins, the grid is giving an
accurate map of the cross-sections. Though the difference of cross-section is not very large (13.9 nB, 2\% of
the total cross-section), it has to be taken into account for a proper event generation.

%----------------------------------------------------------------------------------------

\section{TDJANGOH Interface}

In order to create a C++ class (called TDJANGOH) that plays the role of an interface, I had to modify some
FORTRAN parts of DJANGOH, especially the input method. DJANGOH is working with an input
file where codewords with set values are specified in order to configure the generator.
This was not convenient for the idea of an interface. Thus I have drawn
correspondances between Common Blocks in FORTRAN and structures in C++ so that I can specify
values in the C++ structure and the change is repercuted in the FORTRAN code and vice-versa.
It is useful to specify the values for the input but also to recover the results of the
hadronization that are located in the LUJETS Common Block.
The idea is that within TGEANT the user specifies the input for DJANGOH, the interface
pass it to the generator and the interface recovers the results of the generator and gives
it to TGEANT.
