% Chapter 2

\chapter{Renormalization and QED Radiative Corrections} % Chapter title

\label{ch:Renorm} % For referencing the chapter elsewhere, use \autoref{ch:name}

%----------------------------------------------------------------------------------------

\section{Renormalization in QED}

Content

%----------------------------------------------------------------------------------------

\section{QED Radiative Corrections}

Radiative corrections had a key role in the development of QED : they enable one to calculate cross-sections with extremely high precision
that has until now not been contradicted by experiment.
When a process like DIS involves charged particles, there is a rearrangement of the electromagnetic current between the initial and final state.
Incidentally, real photons can be emitted.

If one were to compute the cross section of the process $1+2 \rightarrow 3+4$ with no photon emission (Born level process), one would find a different result
$\sigma_{0}$ than the measurement $\sigma_{exp}$. To obtain a more accurate result, one has to consider the processes $1+2 \rightarrow 3+4+\gamma_1+\gamma_2+..+\gamma_n$.
Taking these corrections to Born level process into account, one obtains :

\begin{equation} \label{eq:RC}
  \sigma_{exp} = (1+\delta_{RC})\sigma_0
\end{equation}

where $\delta_{RC}$ are the radiative corrections.


In this note, I will not discuss the corrections beyond first order. These first order corrections are also known as order $\alpha$
($o(\alpha)$) corrections. They comprise :
\begin{itemize}
\item Leptonic radiation
\item Hadronic radiation (parton model)
\item Interference of lepton/hadron radiation (two-photon exchange)
\item Vacuum polarization
\item Weak corrections
\end{itemize}

The goal is to quantify the effect of radiative corrections. The observed cross-section can be expressed as the convolution of the true cross-section times a
function called the radiator function which takes into account the radiative effects :

\[d\sigma^{obs}(p,q) = \int \frac{d^{3}k}{2k^{0}}R(l,l',k)d\sigma^{true}(p,-q,k)\]

This relation holds also for the structure functions :

\[F_{n}^{obs}(x,Q^{2}) = \int d\tilde{x}d\tilde{Q}^{2}R_{n}(x,Q^{2},\tilde{x},\tilde{Q}^{2})F_{n}^{true}(\tilde{x},\tilde{Q}^{2})\]

The previous formulas are valid for one photon emission but can be extended to include higher-order
multi-photon emissions.
As one has access to both observed quantities and radiator function, the determination of the true cross-sections
or structure functions from measured ones is done by unfolding using an iterative procedure.
The principal drawbacks of such a method is that the solution is ill-defined : there is no unique solution,
there are large uncertainties and the process is numerically unstable.
However, using partial functioning on the radiator function is reducing the instability of the calculation :

\[R(l,l',k) = \frac{I}{k.l}+\frac{F}{k.l'}+\frac{C}{\tilde{Q}^{2}}\]

The partial functioning is splitting the radiator function in three :
\begin{itemize}
\item Initial state radiation ($I$ fraction)
\item Final state radiation ($F$ fraction)
\item Compton peak ($C$ fraction)
\end{itemize}

For each of them, an observation can be made. For initial state radiation (ISR), $k.l$ is small for
$\angle (l_{in},\gamma) \rightarrow 0$, for final state radiation (FSR), $k.l'$ is small for
$\angle (l_{out},\gamma) \rightarrow 0$ and eventually for Compton peak $Q^{2}$ is small for
$p_{T}(l_{out}) \simeq p_{T}(\gamma)$.

For ISR and FSR, the photon is emitted within narrow cones with a width of the order of $\sqrt{\frac{m_{t}}{E_{t}}}$.
If the particle is massless, the radiated photon will be colinear to the lepton (incoming or
outgoing, whether it is an ISR or a FSR).

Two last notes are to be made :
\begin{itemize}
\item As $E^{2}_{\gamma,max} \propto Q^{2}\frac{1-x}{x}$, the largest radiation in energy are at large $Q^{2}$ and
small $x$. Radiation is suppressed at small $Q^{2}$ and large $x$. There are also large negative
corrections from uncancelled virtual contributions.
\item As $\tilde{Q}^{2}_{min} = \frac{x^{2}}{1-x}M^{2}_{N}$, the case where $\tilde{Q}^{2}_{min} \ll Q^{2}$
is possible.
\end{itemize}

All the preceding explanations did concern leptonic radiation. One should also adress the question of
the hadronic corrections (quark line radiation). These corrections are infrared divergent (radiation of soft photons and gluons)
but they cancel with loops. The emission of the photon/gluon is collinear and gives rise to correction of
type $\frac{\alpha}{2\pi}log(m_{q}^{2})$. Nevertheless for quarks, the approximation $m_{q} \approx 0$ is
giving rise to divergent corrections. One way to solve this issue is to factorize and absorb the divergences
into the PDFs :

\[d\sigma = \sum_{f}d\hat{\sigma}_{f}(1+\delta_{f}(Q^{2},m^{2}_{q}))q_{f}(x) = \sum_{f}d\hat{\sigma}_{f}\hat{q}_{f}(x,Q^{2})\]

\subsection{Characterization and impact of radiative corrections in analysis}

In the following, only DIS/SIDIS is considered.

The description of a radiative event is given by the following : an event is called radiative as soon as it
contains one real radiated photon which is emitted in the lepton line (Fig. \ref{fig:rad_evt}).

\begin{figure}[htb]
\centerline{\epsfig{file=gfx/diagram_rad.png,width=7cm}}
\caption{Typical feynman diagram of a radiative event. One can note that the pair $(Q^2,\nu)$ at the vertex (called hadronic)
is not the same as the one calculated using the incoming and outgoing lepton (called leptonic). The relation between the two pairs
is drawn by :
$\nu_{had} = \nu_{lep} - E_\gamma, Q^2_{had}=Q^2_{lep}+2E_\gamma(\nu_{lep}-\sqrt{\nu_{lep}^2+Q^2_{lep}}cos\theta_\gamma)$}
\label{fig:rad_evt}
\end{figure}


In the following, we will only consider these corrections (Fig. \ref{fig:rad_dia}) :
\begin{itemize}
\item Internal Bremsstrahlung (from both incoming and outgoing leptons) (b,c)
\item Vertex correction (d)
\item Vacuum polarization (e)
\end{itemize}

\begin{figure}[htb]
\centerline{\epsfig{file=gfx/all_rad_feynm.png,width=12cm}}
\caption{List of the diagrams used for the calculation of the radiative corrections. From left to right,
tree level, internal bremsstrahlung (incoming and outgoing leptons), vertex correction and vacuum polarization.}\label{fig:rad_dia}
\end{figure}

Correction to the quark line are not included in any calculation, as explained in Section \ref{sec:RCF}.
If we call $\sigma_{Born}$ the cross-section of the tree-level diagram and $\sigma_{Born+o(\alpha)}$ the
cross-section of tree-level plus the first order correction enumerated above, the definition of the
radiative corrections factor $\eta$ is :

\begin{equation} \label{eq:RCF_def}
  \eta(x,y)=\frac{\sigma_{Born}(x,y)}{\sigma_{Born+o(\alpha)}(x,y)}
\end{equation}

where $x$ is the Bjorken-x aka the fraction of momentum of the incoming proton carried by the struck parton
and $y$ is the fraction of energy of the incoming lepton transferred by the virtual photon.

Observation : the definition of $\eta$ is not fixed and you can find other definition of it.
You can especially encounter the following definition :

\begin{equation} \label{eq:RCF_def_inv}
  \eta(x,y)=\frac{\sigma_{Born+o(\alpha)}(x,y)}{\sigma_{Born}(x,y)}
\end{equation}

which is the inverse of the one before. \textbf{Be extremely careful of the definition of $\eta$ you are using
as it can be very deceptive}. In this note, the former definition (\ref{eq:RCF_def}) will be used.

Obviously, the emission of a real photon is modifying the kinematic variables of the event : two events
will share the same leptonic variables but they will have different hadronic variables. In the case of multiplicities, this discrepancy in the kinematic variables induces that
some hadrons are falling into the wrong (x,y) bin. Applying the correction factor $\eta$ to the multiplicities
is indirectly redirecting the hadrons between (x,y) bins.

\subsection{About emission of radiative photons}

There are two priviledged angles for emission of a real photon :
\begin{itemize}
\item One in the direction of the incident lepton (s-peak)
\item One in the direction of the outgoing lepton (p-peak)
\end{itemize}

\begin{figure}[hbt]
\parbox{8cm}{
\includegraphics[width=9cm]{gfx/plan_angle.png}
\caption{Angles characterizing the emission of a radiative photon : $\theta_\gamma$ is the polar angle and $\Phi_\gamma$.}
\label{fig:plan}}
\qquad
\begin{minipage}{8cm}
\includegraphics[width=8cm]{gfx/peaks.png}
\caption{Location of the priviledged angles in the $(Q^2,\nu)$ plane. The phase space is delimited by the elastic scattering
on top and the boundary values of $cos\theta_\gamma$. The s-peak is located near the boundary $cos\theta_\gamma=1$ which means
$\theta_\gamma \equiv 0[2\pi]$, thus colinear to the incoming lepton.}
\label{fig:peaks}
\end{minipage}
\end{figure}

In the case of muons, note that the s and p peaks are much less pronounced than for electrons.

These knowledge will later be useful to verify the consistency of DJANGOH results.
