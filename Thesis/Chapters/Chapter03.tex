% Chapter 3

\chapter{The COMPASS experiment at CERN} % Chapter title

\label{ch:exp} % For referencing the chapter elsewhere, use \autoref{ch:name}

In this chapter a description of the COMPASS experiment is provided. The general features of the spectrometer
are given in section\ref{}. The beam and target are presented in section\ref{}. The descriptions of the
detectors used for the tracking and the ones used for the particle identification are respectively done
in section\ref{}. The trigger system is discussed in section\ref{}. Eventually the last sections deal with
data acquisition and reconstruction.

%----------------------------------------------------------------------------------------

\section{General Overview}

COMPASS is a high energy, high rate and fixed target experiment at the Super Proton Synchrotron
(SPS) at CERN. It is dedicated to the study of hadron structure and hadron spectroscopy with high
intensity muon and hadron beams.

The COMPASS spectrometer (as in Fig.\ref{}) covers a large kinematic domain : $Q^2$ values up to 100 (GeV/c)$^2$
and x values down to $10^{-5}$ for an incoming muon beam of 160 Gev/c.

The apparatus is divided in three parts : the first part is dedicated to the detection of the incoming beam and
is located upstream the target location. The second and third parts are downstream of the target and are totalizing
a length of 50 meters. The second part called the \textit{Large Angle Spectrometer} (LAS) is built around the magnet SM1.
The LAS has been designed to provide a 180 mrad acceptance. The \textit{Small Angle Spectrometer} (SAS), built around the
magnet SM2, measures the particles emitted at small angles ($\pm$30 mrad).

In 2016, the data taking was performed with a 160 Gev/c muon beam scattering on a liquid H$_2$ target.

%----------------------------------------------------------------------------------------

\section{Beam}

Content

%------------------------------------------------

\subsection{Subsection Title}

Content

%------------------------------------------------

\subsection{Subsection Title}

Content

%----------------------------------------------------------------------------------------

\section{Beam Telescope}

Content

%----------------------------------------------------------------------------------------

\section{Target Region}

Content

%----------------------------------------------------------------------------------------

\section{Spectrometer}

Content

%----------------------------------------------------------------------------------------

\section{Data Acquisition and Online Monitoring}

Content
