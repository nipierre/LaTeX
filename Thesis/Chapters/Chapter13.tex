% Chapter X

\chapter{Determination of quark fragmentation functions} % Chapter title

\label{ch:FF} % For referencing the chapter elsewhere, use \autoref{ch:name}

%----------------------------------------------------------------------------------------

\section{Direct extraction of Fragmentation Functions}

\subsection{Three fragmentation functions case}

Take the relation between the number of hadrons $\frac{dN^h}{dxdz}$, the unpolarized parton distributions $q(x,Q^2)$ and the fragmentation functions $D^h_q(z)$. By normalizing to the number of scattered leptons $\frac{dN^l}{dx}$, the relation gives at LO QCD :

\begin{equation} \label{eq:mult_basic}
  \frac{\frac{dN^h}{dxdQ^2dz}}{\frac{dN^l}{dxdQ^2}} = M^h(x,Q^2,z) = \frac{\sum_{q}e^2_qq(x,Q^2)\D{h}{q}(z))}{\sum_{q}e^2_q(x)}
\end{equation}

where $M^h(x,Q^2,z)$ is the multiplicities defined as in the SIDIS analysis \cite{Multiplicities}.
The sums are running over q = $u,d,s,\ubar,\dbar,\sbar$.

Explicitating the result for respectively proton Eq.\eqref{eq:mult_prot} and deuteron target Eq.\eqref{eq:mult_deut} \cite{Jorg}:

\begin{equation} \label{eq:mult_prot}
  M^h_p = \frac{4u\D{h}{u}+d\D{h}{d}+s\D{h}{s}+4\ubar\D{h}{\ubar}+\dbar\D{h}{\dbar}+\sbar\D{h}{\sbar}}{4(u+\ubar)+(d+\dbar)+(s+\sbar)}
\end{equation}

\begin{equation} \label{eq:mult_deut}
  M^h_d = \frac{(u+d)(4\D{h}{u}+\D{h}{d})+(\ubar+\dbar)(4\D{h}{\ubar}+\D{h}{\dbar})+2s\D{h}{s}+2\sbar\D{h}{\sbar}}{5(u+\ubar+d+\dbar)+2(s+\sbar)}
\end{equation}

Let us fix $h$ = $K^\pm$. In total there are 12 fragmentation functions arising from the expressions of $M^{K^+}$ and $M^{K^-}$. By assuming charge conjugation symmetry ($\D{K^+}{u}$ = $\D{K^-}{\ubar}$, etc.), this number is reduced to 6. These 6 fragmentation functions can then be classified in three groups (e.g. for $K^+$) : the favoured strange quark fragmentation function $\D{K^+}{str} = \D{K^+}{\sbar}$, the favoured up quark fragmentation function $\D{K^+}{fav} = \D{K^+}{u}$ and the unfavoured fragmentation functions $\D{K^+}{unf} = \D{K^+}{s},\D{K^+}{\ubar},\D{K^+}{\dbar},\D{K^+}{d}$. Thus, assuming $\D{K^+}{s} = \D{K^+}{\ubar}$, this leads to the following equations :

\begin{equation} \label{eq:mult_prot_K+}
  M^{K^+}_p = \frac{4u\D{K}{fav}+\sbar\D{K}{str}+(4\ubar+s+d+\dbar)\D{K}{unf}}{4(u+\ubar)+(d+\dbar)+(s+\sbar)}
\end{equation}

\begin{equation} \label{eq:mult_prot_K-}
  M^{K^-}_p = \frac{4\ubar\D{K}{fav}+s\D{K}{str}+(4u+\sbar+d+\dbar)\D{K}{unf}}{4(u+\ubar)+(d+\dbar)+(s+\sbar)}
\end{equation}

\begin{equation} \label{eq:mult_deut_K+}
  M^{K^+}_d = \frac{4(u+d)\D{K}{fav}+2\sbar\D{K}{str}+((u+d)+5(\ubar+\dbar)+2s)\D{K}{unf}}{5(u+\ubar+d+\dbar)+2(s+\sbar)}
\end{equation}

\begin{equation} \label{eq:mult_deut_K-}
  M^{K^-}_d = \frac{4(\ubar+\dbar)\D{K}{fav}+2s\D{K}{str}+((\ubar+\dbar)+5(u+d)+2\sbar)\D{K}{unf}}{5(u+\ubar+d+\dbar)+2(s+\sbar)}
\end{equation}

As we only need three equations to have a Cramer system, we can combine Eq.\eqref{eq:mult_deut_K+} and Eq.\eqref{eq:mult_deut_K-} :

\begin{equation} \label{eq:mult_deut_K}
  M^{K}_d = \frac{4(u+d+\ubar+\dbar)\D{K}{fav}+2(s+\sbar)\D{K}{str}+(6(u+d+\ubar+\dbar)+2(s+\sbar))\D{K}{unf}}{5(u+\ubar+d+\dbar)+2(s+\sbar)}
\end{equation}

Eventually, the following system is obtained :

\begin{equation} \label{eq:mult_sys}
  \begin{split}
  \begin{pmatrix}
    M^{K^+}_p \\
    M^{K^-}_p \\
    M^{K}_d
  \end{pmatrix}
  = \\
  \begin{pmatrix}
    \frac{4u}{4(u+\ubar)+(d+\dbar)+(s+\sbar)} & \frac{\sbar}{4(u+\ubar)+(d+\dbar)+(s+\sbar)} & \frac{4\ubar+s+d+\dbar}{4(u+\ubar)+(d+\dbar)+(s+\sbar)} \\
    \frac{4\ubar}{4(u+\ubar)+(d+\dbar)+(s+\sbar)} & \frac{s}{4(u+\ubar)+(d+\dbar)+(s+\sbar)} & \frac{4u+\sbar+d+\dbar}{4(u+\ubar)+(d+\dbar)+(s+\sbar)} \\
    \frac{4(u+d+\ubar+\dbar)}{5(u+\ubar+d+\dbar)+2(s+\sbar)} & \frac{2(s+\sbar)}{5(u+\ubar+d+\dbar)+2(s+\sbar)} & \frac{6(u+d+\ubar+\dbar)+2(s+\sbar)}{5(u+\ubar+d+\dbar)+2(s+\sbar)}
  \end{pmatrix}
  \begin{pmatrix}
    \D{K}{fav} \\
    \D{K}{str} \\
    \D{K}{unf}
  \end{pmatrix}
  \end{split}
\end{equation}

Lets take a simple model where the nucleon is described by a valence ($q_v$) and a sea quark ($\qbar$) distribution, the quark distribution can be written as \cite{Jorg} :

\begin{center}
  \begin{tabular}{ l }
    $u = 2q_v + \alpha_u\qbar$ \\
    $d = q_v + \alpha_d\qbar$ \\
    $q = \alpha_q\qbar$ for $q=\ubar,\dbar,s,\sbar$ \\
  \end{tabular}
\end{center}

With this model, the previous matrix has the following form :

\begin{equation} \label{eq:mat_rank}
  \begin{pmatrix}
    \frac{8q_v+4\qbar}{9q_v+12\qbar} & \frac{\qbar}{9q_v+12\qbar} & \frac{q_v+7\qbar}{9q_v+12\qbar} \\
    \frac{4\qbar}{9q_v+12\qbar} & \frac{\qbar}{9q_v+12\qbar} & \frac{9q_v+7\qbar}{9q_v+12\qbar} \\
    \frac{12q_v+16\qbar}{15q_v+24\qbar} & \frac{4\qbar}{15q_v+24\qbar} & \frac{18q_v+28\qbar}{15q_v+24\qbar}
  \end{pmatrix}
\end{equation}

Assuming that the PDFs are not pathological, one is assured that the rank of this matrix is 3, then the extraction is feasable.

By performing matrix inversion, the result of $\D{K}{fav}(z)$, $\D{K}{str}(z)$ and $\D{K}{unf}(z)$ can be obtained using data from various $(x,y)$ bins which in fact cover different $Q^2$ ranges.

\subsection{Extension to four fragmentation functions}

As there were initially 4 fragmentation functions, we can adress the question of the assumption
$\D{K^+}{s}=\D{K^+}{\ubar}=\D{K^+}{\dbar}=\D{K^+}{d}$. From PYTHIA simulation, one can find the following
values for the fragmentation functions :

\begin{center}
  \begin{tabular}{ || c | c | c || }
    \hline \hline
    quark $q$ & hadron $h$ & $\int_{0.2}^{1}\D{h}{q}dz$ \\ \hline
    $\sbar$ & $K^+$ & 0.35 \\
    $u$ & $K^+$ & 0.09 \\
    $d$ & $K^+$ & 0.06 \\
    $s$ & $K^+$ & 0.05 \\
    $\ubar$ & $K^+$ & 0.03 \\
    $\dbar$ & $K^+$ & 0.03 \\
    \hline \hline
  \end{tabular}
\end{center}

The table clearly points that for example, $\D{K^+}{d} \approx 2\D{K^+}{\dbar}$. This can be explained
in this case from the fact that $K^+$ may originate from $K^{0*} = (d\sbar) \rightarrow K^+\pi^+$ and
not from $\bar{K}^{0*} \rightarrow K^-\pi^+$.

A new unfavoured fragmentation function can be added by splitting the former unfavoured one into two, viz. :

\begin{center}
  \begin{tabular}{ l }
    $\D{K}{fav}=\D{K^+}{u}$ \\
    $\D{K}{str}=\D{K^+}{\sbar}$ \\
    $\D{K}{unf_1}=\D{K^+}{s}=\D{K^+}{d}$ \\
    $\D{K}{unf_2}=\D{K^+}{\ubar}=\D{K^+}{\dbar}$ \\
  \end{tabular}
\end{center}

Thus, the system is :

\begin{equation} \label{eq:mult_sys4}
  \begin{split}
  \begin{pmatrix}
    M^{K^+}_p \\
    M^{K^-}_p \\
    M^{K^+}_d \\
    M^{K^-}_d \\
  \end{pmatrix}
  = \\
  \begin{pmatrix}
    \frac{4u}{4(u+\ubar)+(d+\dbar)+(s+\sbar)} & \frac{\sbar}{4(u+\ubar)+(d+\dbar)+(s+\sbar)} & \frac{s+d}{4(u+\ubar)+(d+\dbar)+(s+\sbar)} & \frac{4\ubar+\dbar}{4(u+\ubar)+(d+\dbar)+(s+\sbar)} \\
    \frac{4\ubar}{4(u+\ubar)+(d+\dbar)+(s+\sbar)} & \frac{s}{4(u+\ubar)+(d+\dbar)+(s+\sbar)} & \frac{\sbar+\dbar}{4(u+\ubar)+(d+\dbar)+(s+\sbar)} & \frac{4u+d}{4(u+\ubar)+(d+\dbar)+(s+\sbar)} \\
    \frac{4(u+d)}{5(u+\ubar+d+\dbar)+2(s+\sbar)} & \frac{2\sbar}{5(u+\ubar+d+\dbar)+2(s+\sbar)} & \frac{u+d+2s}{5(u+\ubar+d+\dbar)+2(s+\sbar)} & \frac{5(\ubar+\dbar)}{5(u+\ubar+d+\dbar)+2(s+\sbar)} \\
    \frac{4(\ubar+\dbar)}{5(u+\ubar+d+\dbar)+2(s+\sbar)} & \frac{2s}{5(u+\ubar+d+\dbar)+2(s+\sbar)} & \frac{\ubar+\dbar+2\sbar}{5(u+\ubar+d+\dbar)+2(s+\sbar)} & \frac{5(u+d)}{5(u+\ubar+d+\dbar)+2(s+\sbar)}
  \end{pmatrix}
  \begin{pmatrix}
    \D{K}{fav} \\
    \D{K}{str} \\
    \D{K}{unf_1} \\
    \D{K}{unf_2}
  \end{pmatrix}
\end{split}
\end{equation}

Lets take again the same simple model for quark distributions, leading to the following matrix :

\begin{equation} \label{eq:mat_rank_4E}
  \begin{pmatrix}
    \frac{8q_v+4\qbar}{9q_v+12\qbar} & \frac{\qbar}{9q_v+12\qbar} & \frac{q_v+2\qbar}{9q_v+12\qbar} & \frac{5\qbar}{9q_v+12\qbar} \\
    \frac{4\qbar}{9q_v+12\qbar} & \frac{\qbar}{9q_v+12\qbar} & \frac{2\qbar}{9q_v+12\qbar} & \frac{9q_v+5\qbar}{9q_v+12\qbar} \\
    \frac{12q_v+8\qbar}{15q_v+24\qbar} & \frac{2\qbar}{15q_v+24\qbar} & \frac{3q_v+4\qbar}{15q_v+24\qbar} & \frac{10\qbar}{15q_v+24\qbar} \\
    \frac{8\qbar}{15q_v+24\qbar} & \frac{2\qbar}{15q_v+24\qbar} & \frac{4\qbar}{15q_v+24\qbar} & \frac{15q_v+10\qbar}{15q_v+24\qbar}
  \end{pmatrix}
\end{equation}

Assuming that the PDFs are not pathological, one is assured that the rank of this matrix is 4, then the extraction
is feasable.

%----------------------------------------------------------------------------------------

\section{COMPASS LO fit}

Content
