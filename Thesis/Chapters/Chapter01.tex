% Chapter 1

\chapter{Theoretical framework} % Chapter title

\label{ch:thfw} % For referencing the chapter elsewhere, use \autoref{ch:name}

%----------------------------------------------------------------------------------------

\section{Deep Inelastic Scattering}

The Deep Inelastic Scattering (DIS) process is depicted in Fig.\ref{pic:DIS}. The incoming lepton $l$ exchanges
a virtual photon $\gamma^*$ with the nucleon $N$. The nucleon absorbs the energy of the virtual photon and fragments
into a final state X. The scattered lepton is represented by $l'$. This process description is also known as the
one photon exchange approximation.

% Pic DIS

The kinematics of a DIS event are fixed by the 4-momentum vector of l (\textbf{l} = ($E$,$\vec{l}$)), $l'$ (\textbf{l'} = ($E'$,$\vec{l}'$))
and N (\textbf{P} = ($M$,$\vec{0}$)). The 4-momentum vector for the virtual photon is calculated as \textbf{q} = \textbf{l} - \textbf{l'} =
($\nu$ = $E$ - $E'$, $\vec{q}=\vec{l}-\vec{l}'$). One needs only two Lorentz invariant variables to describe
inclusive DIS. One is the invariant mass of the virtual photon $Q^2$.

\begin{equation}
  Q^2 = -\textbf{q}^2 \stackrel{lab}{\approx} 4EE'sin^2(\frac{\theta}{2})
\end{equation}

$Q^2$ defines the scale at which the nucleon structure is probed : the largest $Q^2$ is, the farthest the probing
of the nucleon is performed.

The other is $x$ and measures the elasticity of the interaction. $x$ is comprised between 0 and 1. If $x=1$ the
interaction is elastic, if $x<1$ then it is inelastic. This Bjorken\cite{Bjorken} $x$ is defined as the fraction
of the nucleon 4-momentum \textbf{P} carried by the parton struck by the virtual photon.

\begin{equation}
  x = \frac{Q^2}{2\textbf{P}\cdot\textbf{q}} \stackrel{lab}{=} \frac{Q^2}{2M\nu}
\end{equation}

Other Lorentz invariant can be as well (Table\ref{tab:kinvar})

\begin{tabular*}{\textwidth}{r|l}
  \hline
  \hline
  Variable & Description \\
  \hline
  \hline
  $Q^2 = -\textbf{q}^2 \stackrel{lab}{\approx} 4EE'sin^2(\frac{\theta}{2})$ & Interaction scale \\
  $\nu = \frac{\textbf{P}\cdot\textbf{q}}{M} \stackrel{lab}{=} E - E'$ & Energy transfer from the lepton $l$ to $\gamma^*$ \\
  $x = \frac{Q^2}{2\textbf{P}\cdot\textbf{q}} \stackrel{lab}{=} \frac{Q^2}{2M\nu}$ & Fraction of the nucleon momentum \textbf{P} carried by the parton struck by $\gamma^*$ \\
  $\nu = \frac{\textbf{P}\cdot\textbf{q}}{\textbf{P}\cdot\textbf{l}} \stackrel{lab}{=} \frac{\nu}{E}$ & Fraction of the incoming lepton energy transferred to $\gamma^*$ \\
  $s = (\textbf{P}+\textbf{l})^2 \stackrel{lab}{\approx} M^2 + 2ME$ & Center-of-mass energy squared \\
  $W^2 = (\textbf{P}+\textbf{q})^2 \stackrel{lab}{=} M^2 + 2M\nu - Q^2)$ & Invariant mass of the hadronic final state \\
  \hline
  \hline
\end{tabular*}

\subsection*{Cross section calculation for the inclusive DIS process}

The deep inelastic cross section, in the one photon exchange approximation, can be written in terms
of the lepton-photon coupling tensor $L_{\mu\nu}$ and the hadronic coupling tensor $W^{\mu\nu}$.

\begin{equation}
  \frac{d\sigma}{dE'd\Omega} = \frac{\alpha^2}{2Mq^4}\frac{E'}{E}L_{\mu\nu}W^{\mu\nu}
\end{equation}

$\alpha$ is the fine structure constant. The leptonic and hadronic tensors are defined by :

\begin{equation}
  L_{\mu\nu}(l,s;l') = 2{L^{(S)}_{\mu\nu}(l;l')+iL^{(A)}_{\mu\nu}(l,s;l')}
\end{equation}

where

\begin{equation}
  L^{(S)}_{\mu\nu} = l_{\mu}'l_{\nu} + l_{\nu}'l_{\mu} - g_{\mu\nu}(\vec{l}'\vec{l}-m^2) \\
  L^{(A)}_{\mu\nu} = -m\epsilon_{\mu\nu\sigma\rho}s^{\sigma}q^{\rho}
\end{equation}

and

\begin{equation}
  W^{\mu\nu}(q;P,s) = W^{\mu\nu\ (S)}(q;P) + iW^{\mu\nu\ (A)}(q;P,S)
\end{equation}

where

\begin{equation}
  \frac{1}{2M}W^{\mu\nu\ (S)}(q;P) = W_1(P\cdot q,q^2)(-g^{\mu\nu}-\frac{q^{\mu}q^{\nu}}{q^2})+\frac{W_2(P\cdot q,q^2)}{M^2}(P^{\mu}-\frac{P\cdot q}{q^2}q^{\mu})(P^{\nu}+\frac{P\cdot q}{q^2}q^{\nu})
\end{equation}

\begin{equation}
  \frac{1}{2M}W^{\mu\nu\ (A)}(q;P,S) = \epsilon_{\mu\nu\sigma\rho}q^{\sigma}{G_1(P\cdot q,q^2)MS^{\rho}+\frac{G_2(P\cdot q,q^2)}{M}(P\cdot q)S^{\rho}-(S\cdot q)P^{\rho}}
\end{equation}

The lepton and nucleon polarizations are given respectively by $s$ and $S$. $g_{\mu\nu}$ is the
Minkowski metric and m is the lepton mass. The functions $W_1(P\cdot q,q^2)$, $W_2(P\cdot q,q^2)$,
$G_1(P\cdot q,q^2)$ and $G_2(P\cdot q,q^2)$ are the spin averaged and spin dependent structure functions
parametrizing the internal structure of the nucleon. They can be expressed as dimensionless functions as
in Eqs.\ref{} to \ref{}.

\begin{equation}
  MW_1(P\cdot q,Q^2)=F_1(x,Q^2)
  \nu W_2(P\cdot q,Q^2)=F_2(x,Q^2)
  \frac{(P\cdot q)^2}{\nu}G_1(P\cdot q,Q^2)=g_1(x,Q^2)
  \nu(P\cdot q)G_2(P\cdot q,Q^2)=g_2(x,Q^2)
\end{equation}

Going back to Eq.\ref{} and splitting the symmetric and antisymmetric parts of the tensors :

\begin{equation}
  \frac{d\sigma}{dE'd\Omega} = \frac{\alpha^2}{2Mq^4}\frac{E'}{E}[L_{\mu\nu\ (S)}W^{\mu\nu\ (S)}-L_{\mu\nu\ (A)}W^{\mu\nu\ (A)}]
\end{equation}

After summing over all possible spin configurations of the initial and final lepton and nucleon, neglecting
the leptonic mass, one obtains the unpolarized DIS cross-section (Eq.\ref{}) which measurement gives access
to the structure functions $F_1$ and $F_2$.

\begin{equation}
  \frac{d\sigma^{unpolarized}}{dxdQ^2} = \frac{4\pi\alpha}{Q^4}[y^2F_1(x,Q^2)+(\frac{1-y}{x}-\frac{My}{2E})F_2(x,Q^2)]
\end{equation}

%------------------------------------------------

\section{Quark Parton Model}

The Quark Parton Model is developed in the infinite momentum frame where the nucleon has a very large momentum
along a certain direction and is composed by massless point-like particles called partons. In this case, the
transverse momentum of these partons can be neglected. In DIS, the virtual photon interacts with the parton,
which carries a fraction $\xi$ of the 4-momentum \textbf{P} of the nucleon and the invariant mass of the initial
and final states are respectively $(\xi\textbf{P}+\textbf{q})^2$ and 0. This yelds :

\begin{equation}
  (\xi\textbf{P}+\textbf{q})^2 = 0 \Rightarrow 2\xi\textbf{P}\cdot\textbf{q}+\textbf{q}^2 = 0 \Rightarrow \xi = \frac{Q^2}{2\textbf{P}\cdot\textbf{q}}
\end{equation}

which is the exact definition of Bjorken $x$.

Within this model, since gluons do not carry any electric charge, the DIS interaction can only involve quarks and
it has to be noted that the surrounding quarks are not affected by the interaction. The hadronic tensor is defined
by Eq.\ref{}.

\begin{equation}
  W^{\mu\nu} = \sum\limits_{q,s}e^2_qn_q(x,s;S)\frac{1}{\textbf{P}\cdot\textbf{q}}[2x\textbf{P}^{\mu}\textbf{P}^{\nu}
  +\textbf{P}^{\nu}\textbf{q}^{\mu}+\textbf{P}^{\mu}\textbf{q}^{\nu}-g^{\mu\nu}\textbf{P}\cdot\textbf{q}]
\end{equation}

where $n_q(x,s;S)$ is the density of quarks q with charge $e_q$ and spin $s$, the nucleon spin being given by S.
In this model, the structure functions expressions are :

\begin{equation}
  F_1(x)=\frac{1}{2}\sum\limits_{q}e^2_qq(x) \\
  F_2(x)=x\sum\limits_{q}e^2_qq(x)
\end{equation}

where $q(x)$ are the unpolarized Parton Distribution Functions (PDFs). The sums run over all quark flavors.
One can draw a trivial relationship between $F_1$ and $F_2$, called the Callan-Gross relation :

\begin{equation}
  F_1(x)=\frac{1}{2x}F_2(x)
\end{equation}

As the QPM is considered inside the infinite momentum frame, $Q^2 \rightarrow \infty$ and the $Q^2$ dependence of
$F_1$ and $F_2$ is thus lost : this sclaing violation is also known as the Bjorken scaling. This observation along
with the Callan-Gross relation showed that partons are indeed 1/2-spin particles.

Eventually yields the unpolarized DIS cross-section with the QPM :

\begin{equation}
  \frac{d^2\sigma^{unpolarized}}{dxdy} \stackrel{QPM}{=} \frac{8\pi\alpha^2ME}{Q^2}[\frac{1}{2}y^2+(1-y-\frac{y^2\gamma^2}{4})]x\sum\limits_{q}e^2_qq(x)
\end{equation}

\subsection*{Scaling violation}

The structure function $F_2$ has been measured by several collaborations covering a wide $x$ - $Q^2$ kinematic range.
The measured values are plotted as a function of $Q^2$ and in bins of $x$. The Bjorken scaling is only visible in a
small $x$ region between 0.1 and 0.4. Outside this region the structure function $F_2$ has a logarithmic dependence on
$Q^2$. At small $x$, $F_2$ increases with $Q^2$ while at large $x$, $F_2$ decreases for large $Q^2$ values. From the scaling
violation a conclusion was made that there should be a missing piece in the QPM : the gluon contribution. In order to take
into account this contribution, the QPM was developed within the Quantum ChromoDynamics frame (QCD).

%Pic F2 scaling violation

\subsection*{QCD-improved QPM}

The $Q^2$ dependence observed in previous subsection can be estimated by introducing the quark interaction from QCD.
Quantum ChromoDynamics is a non-abelian gauge theory based on a symmetry group SU(3) which describes the strong interaction.
The charge of this theory is called colour and the force carriers are the gluons, which are also coloured particles. The
internal nucleon dynamic is dominated by the gluon emission and absorption by the 3 valence quarks. The gluons can create
pairs of quark-antiquark or emit gluons. This creates a cloud of gluons and virtual $q\bar{q}$ pairs known as sea quarks.

The QCD coupling constant $\alpha_s$ depends on the scale of the interaction. At low energies quarks or gluons are always
forming colorless particles as hadrons : this is the confinement. At high energies quarks or gluons are free particles :
this is asymptotic freedom.

Depending on the energy regime, a process can be labeled as a hard ($\alpha \sim 0$) or soft process ($\alpha \approx 0$).
Hard processes can be described within the perturbative QCD (pQCD) framework when soft processes can only be parametrized
from experimental data. This two regimes differ by the factorisation scale $\Lambda$. As in DIS the scale variable is $Q^2$,
the DIS cross-section can only be factorized in terms of soft and hard processes at $Q^2>1$GeV$^2$ : the hard process is the
lepton-quark interaction $\sigma_q$ and the soft process is a non-calculable universal quantity which are the PDFs.

The resolution of the virtual photon probe is proportional to $1/Q^2$ (Fig.\ref{}). At $Q^2 \approx 0$, the virtual photon sees the
nucleon as a point-like particle. As $Q^2$ increases, the virtual starts to interact with the nucleons constituents. At large $Q^2$
the virtual photon is able to access the gluons. The first QCD correction to the QPM concerns the gluon emission by the initial and
the final quark. The splitting functions $P_{ij}(x/\xi)$ are the probability that a quark or gluon of type $j$ and momentum fraction
$\xi$ is the parent of i with momentum fraction $x$.

% pic res photon

The $Q^2$ dependence is calculated by the Dokshiter-Gribov-Lipatov-Altarelli-Parisi (DGLAP) equations.

\begin{equation}
  \frac{dq_i(x,Q^2)}{dlnQ^2} = \frac{\alpha_s(Q^2)}{2\pi}\sum\limits_{j}\int_{x}^{1}P_{ij}(x/\xi,\alpha_s(Q^2))q_i(\xi,Q^2)
\end{equation}

If the PDFs are known at a given scale $Q_0^2$, they can be evolved to any given $Q^2$ using these equations. 

%----------------------------------------------------------------------------------------

\section{Hadronisation}

Content

%----------------------------------------------------------------------------------------

\section{Measurement of Fragmentation Functions}

Content
